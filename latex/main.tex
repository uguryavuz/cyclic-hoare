\documentclass[10pt]{article}
\usepackage[margin=1.2in]{geometry}

\usepackage{xcolor}
\usepackage{amsmath}  % Align math mode
\usepackage{amssymb}  % Symbols
\usepackage{amsthm}   % Numbered thm/def environments
\usepackage{ebproof}  % Inference rules
\usepackage{enumitem} % Enumeration
\usepackage{graphicx} % Graphics
\usepackage{hyperref} % Hyperlinks
\usepackage{mathpazo} % Palatino - was there ever any other choice?
\usepackage{mathtools}
\usepackage{stmaryrd}
\usepackage{setspace} % Line spacing
\usepackage{titlesec}
\usepackage{mathpartir}

\usepackage{mystyle}

\titleformat{\subsubsection}[runin]
  {\normalfont\normalsize\bfseries}{\thesubsubsection}{1em}{}

% Numbered environments
\theoremstyle{definition}
\newtheorem{definition}{Definition}
\newtheorem{example}{Example}
\newtheorem{theorem}{Theorem}
\newtheorem{conjecture}{Conjecture}
\newtheorem{lemma}{Lemma}
\newtheorem*{theorem*}{Theorem}
\newtheorem{corollary}{Corollary}[theorem]

% \title{Cyclic-Proj}
% \author{Ugur Yavuz}
% \date{May 2024}

\begin{document}


\section{Rule graphs}

\newcommand*{\DomLiftable}{\mathit{Lift}}
\newcommand*{\DomStmt}{\mathit{Stmt}}
\newcommand*{\DomRule}{\mathit{Rule}}
\newcommand*{\LIFT}{\textsc{Lift}}

Suppose we have a family of statements $\DomStmt$,
where for each statement $S \in \DomStmt$
we have a defined notion of validity,
denoted $\VAL S$.
%
We also have a subset of \emph{liftable} statements
$\DomLiftable \subseteq \DomStmt$;
these are statements for which
we are ``comfortable'' lifting a proof of their derivability
to a metalogical proof of their validity.
%
We also have a family of rules $\DomRule$ of the form
\[
    \inferrule*[right=Rule-Name]{
        \DER S_1 \\ \cdots \\ \DER S_n
    }{
        \DER C
    }
\]
%
\begin{definition}
    A rule with premises $S_1,\dots,S_n$ and conclusion $C$ is \emph{locally sound} if
    \[\VAL S_1 \wedge \cdots \wedge \VAL S_n \implies \VAL C.\]
\end{definition}
%

\newcommand*{\DomNode}{\mathit{Nodes}}
\newcommand*{\node}{\mathit{node}}
\newcommand*{\MapConc}{\mathit{conc}}
\newcommand*{\MapRule}{\mathit{rule}}
\newcommand*{\MapPrems}{\mathit{prems}}

\begin{definition}
A \emph{rule graph} is a tuple $G=(\DomNode,\MapConc,\MapRule,\MapPrems)$ for a
finite set of node names $\DomNode$,
and mappings %from node names to statements 
    $\MapConc : \DomNode \ra \DomStmt$,
%a mapping from node names to rule names 
    $\MapRule : \DomNode \ra \DomRule + \LIFT$,
%and a mapping from node names to premises 
 and   $\MapPrems : \DomNode \ra \DomNode^*$,
such that for every $\node \in \DomNode$:
\begin{itemize}
\item If $\MapRule(\node) = r \in \DomRule$
    where $\abs{p} = n$, then
    \[
        \inferrule{
            \DER \MapConc(\MapPrems_1(\node))
            \\ \cdots \\
            \DER \MapConc(\MapPrems_n(\node))
        }{ \DER \MapConc(\node) }
    \]
    is a syntactically valid instance of $r$,\\
    where $\MapPrems_i(\node)$
    denotes $\pi_i(\MapPrems(\node))$.
\item Otherwise, if $\MapRule(\node) = \LIFT$,
    then $\MapPrems(\node) = []$ and 
    $\MapConc(\node) \in \DomLiftable$.
\end{itemize}
\end{definition}

A statement is \emph{in} a graph if $S \in \mathrm{Im}(G.\MapConc)$.
Denote with $S \in G$.

\begin{definition}
    A \emph{path} in a rule graph $(\DomNode,\MapConc,\MapRule,\MapPrems)$
    is a possibly infinite sequence of nodes
    $\node_1,\node_2,\dots$
    all in $\DomNode$
    such that
    $\forall i > 0.\
    \node_{i+1} \in \MapPrems(\node_i).
    $
\end{definition}

\newcommand*{\reaches}[3]{%
    \ensuremath{#1 \leadsto_{#3} #2}%
}
For a rule graph $G$ with a node set $\DomNode$,
and a pair of nodes $\node, \node' \in \DomNode$,
we will let \reaches{\node}{\node'}{G} denote
the existence of a path in $G$
that begins with $\node$ and ends with $\node'$.
Note how a path containing a liftable node 
which employs the \LIFT\ rule must end at that node.

\begin{definition}
    A \emph{(simple) cycle} is a finite path $\node_1,\dots,\node_n$
    for $n>1$,
    where $\node_1 = \node_n$ and all other nodes are distinct.
    %
    A rule graph is \emph{cyclic}
    iff it contains a path which is a cycle.
\end{definition}


\begin{lemma}
    \label{lem:cyclic-iff-no-longest-path}
    A rule graph is cyclic iff
    it does not have a longest path.
\end{lemma}
\begin{proof}
First direction:
Consider any path in the graph.
We can always produce a longer path by iterating over the graph's cycle a sufficient number of times.

Second direction:
If there is no longest path, then there is a path longer than the cardinality of $\DomNode$. By pigeonhole w.r.t.\ the finitude of $\DomNode$,
this path must contain a duplicate node.
A path containing a duplicate node must contain a cycle.
%See Coq. (Depends on finitude of $\DomNode$)
\end{proof}

\newcommand{\restrict}[2]{%
    #1 {\mid_{#2}} %
}

\iffalse
\begin{definition}
    The \emph{subgraph} of a graph $G=(\DomNode,\MapConc,\MapRule,\MapPrems)$ w.r.t.\ $\node \in \DomNode$
    is 
    \\
    $G'=(
        \DomNode',\
        \restrict\MapConc{\DomNode'},\
        \restrict\MapRule{\DomNode'},\
        \restrict\MapPrems{\DomNode'}
    )$,
    where $\DomNode' = 
        \SetBuild(\node' \in \DomNode | \reaches{\node}{\node'}{G})$.
\end{definition}

\begin{lemma}
Any subgraph is well-defined according to the definition of graph.
\end{lemma}
\fi

\subsection{Soundness of acyclic graphs}

\begin{definition}
$(\DomNode,\MapConc,\MapRule,\_)$ is \emph{lift-valid}
if
\(
    \forall \node \in \DomNode.\ \MapRule(\node) = \LIFT \implies
    \VAL \MapConc(\node).
\)
\end{definition}

\begin{definition}
    A graph is \emph{locally sound} if
    all rules in the graph are locally sound.
\end{definition}

\begin{lemma}
    \label{lem:acyclic-soundness}
    If a graph $G$ is lift-valid, locally sound, and acyclic,
    then $\VAL S$ for every $S \in G$.
    %
    %Consider a graph $G=(\DomNode,\MapConc,\MapRule,\MapPrems)$.
    %and a node $\node \in \DomNode$.
    %If $G$ is acyclic, lift-valid, and locally sound,
    %and all rules in the image of $\MapRule$ are locally sound,
    %then $\VAL \MapConc(\node)$ for every node $\node \in \DomNode$.
\end{lemma}

\begin{proof}
We induct over the depth of nodes in the rule graph.
The depth of a leaf is 1;
the depth of a non-leaf is one greater than the maximum depth of all premises.
From local soundness, we know that the validity of a conclusion
comes from the validity of its premises. 
We also know that the depth of a conclusion is greater than that of its premises.
Thus by strong induction on depth,
where the base case is every leaf
(which is either valid by lack of premises or by being lifted),
we find that every statement in the graph must be valid.

\end{proof}

Consider how depth is only well-defined for an acyclic graph.
This means computing depth in our Coq implementation required
proving that depth always exists as a finite value for each node in an acyclic graph.
We implemented depth as a fixpoint with fuel, where the initial fuel is the cardinality of $\DomNode$.
We then had to show that this amount of fuel is sufficient. We argue that if an amount of fuel $f$ is insufficient for a node $n$,
then a path from $n$ with length $> f$ must exist.
Thus, if $\abs\DomNode$ is insufficient fuel for a node,
then a path longer than $\abs\DomNode$ exists,
which is impossible in an acyclic graph.

\iffalse
\begin{lemma}
    Consider a graph $G=(\DomNode,\MapConc,\MapRule,\MapPrems)$,
    and a node $\node \in \DomNode$.
    If the subgraph of $G$ w.r.t.\ $\node$ is acyclic and valid,
    and all rules in the image of $\MapRule$ are locally sound,
    then $\VAL \MapConc(\node)$.
\end{lemma}
\fi

\newcommand{\DERby}[2]{\DER_{#1}{#2}}

\begin{definition}
    $S$ is \emph{derivable via graph} $G$, denoted $\DERby G S$,
    if $S \in G$ and $G$ is lift-valid.\\
    $S$ is \emph{derivable}, denoted $\DER S$,
    if a graph $G$ exists s.t.\ $\DERby G S$.
    %if a lift-valid graph $(\_, conc, \_, \_)$ exists
    %which is lift-valid, and
    %where $S \in \text{Im}(conc)$.
    %contains $\node$
    %where $\MapConc(\node) = S$.
\end{definition}

\begin{theorem}
    For a statement $S$,
    if $\DERby G S$ for an acyclic and locally sound graph $G$, %valid acyclic graph,
    %for which all contained rules are locally sound,
    then $\VAL S$.
\end{theorem}

\begin{proof}
    Follows from Lemma~\ref{lem:acyclic-soundness}.
\end{proof}


\subsection{Completeness}

\begin{definition}
    Suppose we have an oracle $O$ s.t.\
    for any $S \in \DomLiftable$,
    $O$ accepts $S$ iff $\VAL S$.
    A proof system is \emph{relatively complete} w.r.t.\
    the completeness of the metalogical proofs of lifted statements
    %$\DomLiftable$ if,
    if, for any $S \in \DomStmt$ where $\VAL S$,
    a locally sound $G$ exists where $S \in G$
    and $O$ accepts every lifted statement in $G$.
\end{definition}

\subsection{Other stuff}

\begin{definition}
    A graph $(\DomNode,\MapConc,\MapRule,\MapPrems)$
    is \emph{normal} if
    $\MapConc$ is injective.
\end{definition}

Normalization procedure for DAG:
\begin{enumerate}[label=(\roman*)]
    \item Topologically sort the DAG.
    \item If every node is unique, halt. Otherwise:
    \item Consider node $n$ for which there are (at least) two instances.
        Take all edges pointing into the earlier $n$ and redirect them into the latter $n$.
    \item While nodes of indegree 0 exist, remove them.
    \item Return to step (ii).
\end{enumerate}

\begin{conjecture}
If 
\end{conjecture}

\section{Operational semantics of the \textsc{While} language}

\subsection*{Syntax}

\begin{displaymath}
\begin{array}{lrcl}
    & n & \in & \Z \\
    & x & \in  & \domvar \\
    \domvar & v & \bnf & n \ALT \True \ALT \False \\
    \DomAexp & a & \bnf & x \ALT n \ALT a + a \ALT a - a \ALT a \times a \\
    \DomBexp & b & \bnf & \True \ALT \False \ALT 
        \neg b \ALT b \wedge b \ALT b \vee b \ALT b \Rightarrow b \\\
    && \ALT & a = a \ALT b = b \ALT a < a \ALT a \leq a \\
    & e & \bnf & a \ALT b \\ 
    \DomComm & c & \bnf & \Skip \ALT \Assn x = a
        \ALT \Seq c ; c \\
    && | & \ITE b then c else c \\
    && | & \While b do c
\end{array}
\end{displaymath}

\subsection*{Memories and expressions}

Memories (in the family $\Mems$)
are total maps from variables (\domvar) to integers.
%
We define the \emph{evaluation} of an expression $e$
w.r.t.\ a memory $m$, denoted $\Eval(e,m)$,
to yield exactly one value $v \in \Vals$.
%
{
\newcommand{\LINE}[2]{
    \Eval(#1,m) &\triangleq& #2 \\
}
%
\begin{mathpar}
\begin{array}{rcl}
\LINE x {m(x)}
\LINE n n
\LINE \True \True
\LINE \False \False
\LINE {\neg b} 
    {\neg \Eval(b,m)}
\LINE {b_1 \wedge b_2} {
    \Eval(b_1,m) \wedge \Eval(b_2,m)}
\LINE {b_1 \vee b_2} {
    \Eval(b_1,m) \vee \Eval(b_2,m)}
\LINE {b_1 \Rightarrow b_2} {
    \Eval(b_1,m) \Rightarrow \Eval(b_2,m)}
\end{array}
\and
\begin{array}{rcl}
\LINE {a_1 = a_2} {
    \Eval(a_1,m) = \Eval(a_2,m)}
\LINE {b_1 = b_2} {
    \Eval(b_1,m) = \Eval(b_2,m)}
\LINE {a_1 < a_2} {
    \Eval(a_1,m) < \Eval(a_2,m)}
\LINE {a_1 \leq a_2} {
    \Eval(a_1,m) \leq \Eval(a_2,m)}
\LINE {a_1 + a_2} {
    \Eval(a_1,m) + \Eval(a_2,m)}
\LINE {a_1 - a_2} {
    \Eval(a_1,m) - \Eval(a_2,m)}
\LINE {a_1 \times a_2} {
    \Eval(a_1,m) \times \Eval(a_2,m)}
\end{array}
\end{mathpar}

}

%Define the support of an expression $e$ as
%$\Support e \triangleq \SetBuild(m | {\Eval(e,m) \neq \bot})$.

%A \emph{boolean expression} is an expression $P$ s.t.
%$\forall m.\ \Eval(P,m) \in \set{\True,\False,\bot}$.
%Call this family of expressions $\BoolExps$.
%

% \[
%     \begin{prooftree}[center=false]
%     \hypo{\Eval(e,m)=v}
%     \infer1{\Cfg(\Assn x = e, m) \sstepsto \Cfg(\Skip, m)}
%     \end{prooftree}
%     \qquad
%     \begin{prooftree}[center=false]
%     \hypo{ \Gamma, A \vdash B }
%     \infer1{ \Gamma \vdash A \to B }
%     \end{prooftree}
% \]


% \subsection{Memories}

\subsection*{Small-step operational semantics}

\begin{mathpar}
    % Assignment
    \inferrule*[right=SS-Assn]
    { \Eval(a,m)=v }
    { \Cfg(\Assn x = a, m) \sstepsto \Cfg(\Skip, \MemDef m[x->v]) }
    % Sequencing
    \\ \inferrule*[right=SS-Seq]
    { \Cfg(c_0,m) \sstepsto \Cfg(c'_0,m') }
    { \Cfg(\Seq c_0 ; c_1, m) \sstepsto \Cfg(c'_0 ; c_1, m')}
    % Sequencing, skip elimination
    \and \inferrule*[right=SS-Seq-Skip]
    { }
    { \Cfg(\Seq \Skip ; c, m) \sstepsto \Cfg(c, m) }
    % If true
    \and \inferrule*[right=SS-If-T]
    { \Eval(b,m) = \True}
    { \Cfg(\ITE b then c_1 else c_2, m) \sstepsto \Cfg(c_1, m)}
    % If false
    \and \inferrule*[right=SS-If-F]
    { \Eval(b,m) = \False}
    { \Cfg(\ITE b then c_1 else c_2, m) \sstepsto \Cfg(c_2, m)}
    % While true
    \and \inferrule*[right=SS-While-T]
    { \Eval(b,m) = \True}
    { \Cfg(\While b do c, m) \sstepsto \Cfg(\Seq c; \While b do c, m)}
    % While false
    \and \inferrule*[right=SS-While-F]
    { \Eval(b,m) = \False}
    { \Cfg(\While b do c, m) \sstepsto \Cfg(\Skip, m)}
\end{mathpar}

\subsection*{Multi-step reduction}

$\Cfg(c,m) \sstepston n \Cfg(c',m')$ iff
$\Cfg(c_0,m_0),\dots,\Cfg(c_n,m_n)$
s.t.\ $\Cfg(c,m)=\Cfg(c_0,m_0)$,
$\Cfg(c_k,m_k)=\Cfg(c',m')$,
and $\forall i \in [0,n).\ \Cfg(c_i,m_i) \sstepsto \Cfg(c_{i+1},m_{i+1})$.


Let $\sstepstotrans$ be the reflexive transitive closure of $\sstepsto$. Namely,
$\Cfg(c,m) \sstepstotrans \Cfg(c',m')$ iff
$\exists n.\ \Cfg(c,m) \sstepston n \Cfg(c',m')$.

\begin{definition}
    $c$ \emph{yields} $m'$ from $m$,
    denoted $\Yields(c,m,m')$,
    if $\Cfg(c,m) \sstepstotrans \Cfg(\Skip,m')$.
    $c$ \emph{terminates} on $m$, denoted $\Terminate(c,m)$,
    iff $\exists m'. \Yields(c,m,m')$.
    $c$ \emph{diverges} on $m$, denoted $\Diverge(c,m)$,
    iff $\neg \Terminate(c,m)$.
\end{definition}

\begin{lemma}[Sequential execution]
    If $\Cfg(c_1; c_2, m) \sstepston n \Cfg(\Skip, m'')$, 
    then intermediate memory $m'$ exists s.t.\
    $\Cfg(c_1, m) \sstepston{n_1} \Cfg(\Skip, m')$
    and $\Cfg(c_2, m') \sstepston{n_2} \Cfg(\Skip, m'')$
    where $n_1+n_2+1=n$.
\end{lemma}
\begin{proof}
    By induction on length of execution. See Coq proof.
\end{proof}

\begin{lemma}
    If $\Cfg(c,m) \sstepston n \Cfg(c_1,m_1)$,
    and $\Cfg(c,m) \sstepston n \Cfg(c_2,m_2)$,
    then $c_1 = c_2$ and $m_1 = m_2$.
\end{lemma}

\begin{lemma}
    If $\Cfg(c,m) \sstepston{n_1} \Cfg(\Skip,m_1)$,
    and $\Cfg(c,m) \sstepston{n_2} \Cfg(\Skip,m_2)$,
    then $m_1 = m_2$ and $n_1 = n_2$.
\end{lemma}

\section{Assertions}

\newcommand*{\domivar}{\mathsf{IVar}}

\newcommand*{\DomAssert}{\mathsf{Assrt}}
\newcommand*{\DomAexpv}{\mathsf{Aexpv}}

We define boolean assertions as an extension
on boolean expressions.
These allow for quantification over
integer variables.
\begin{displaymath}
\begin{array}{lrcl}
    & i & \in  & \domivar \\
    \DomAexpv & a & \bnf & x \ALT n \ALT i \ALT a + a \ALT a - a \ALT a \times a \\
    \DomAssert & b & \bnf & \True \ALT \False \ALT 
        \neg b \ALT b \wedge b \ALT b \vee b \ALT b \Longrightarrow b \\\
    && \ALT & a = a \ALT b = b \ALT a < a \ALT a \leq a \\
    && \ALT & \forall i.\, b \ALT \exists i.\, b
\end{array}
\end{displaymath}
where $i$ are bound integer variables.
Define a \emph{binding} $I$ as a total mapping $\domivar \rightarrow \Z$.
%
Note how we may implicitly lift any
arithmetic expression $a \in \DomAexp$ into
an expression in $\DomAexpv$,
and similarly a
boolean expression $b \in \DomBexp$ into
an assertion in $\DomAssert$, which
contains no integer variables.

\def\AVAL#1|#2|-#3{%
    #1 \vDash_{#2} #3 %
}
\def\NAVAL#1|#2|-#3{%
    #1 \nvDash_{#2} #3 %
}

\def\AVALmi#1{%
    \AVAL m|I|- #1 %
}

\def\AVALU#1{%
    \vDash #1 %
}

\def\EvalI(#1,#2,#3){%
    \Interp{#1}_{#2,#3}
}

Define evaluation of arithmetic expressions with integer variables:
%
{
\newcommand{\LINE}[2]{
    \EvalI(#1,m,I) &\triangleq& #2 \\
}
%
\begin{mathpar}
\begin{array}{rcl}
\LINE {n} {
    n}
\LINE {x} {
    m(x)}
\LINE {i} {
    I(i)}
\end{array}
\and
\begin{array}{rcl}
\LINE {a_1 + a_2} {
    \EvalI(a_1,m,I) + \EvalI(a_2,m,I)}
\LINE {a_1 - a_2} {
    \EvalI(a_1,m,I) - \EvalI(a_2,m,I)}
\LINE {a_1 \times a_2} {
    \EvalI(a_1,m,I) \times \Eval(a_2,m,I)}
\end{array}
\end{mathpar}
}


The relation $\AVAL m|I|-P$
says that memory $m$ \emph{satisfies} assertion $P$
with binding $I$.
Notation $\AVALU P$ says that $P$ always holds
in any memory and binding.
The inductive rules for \emph{satisfaction} are:
%
\begin{mathpar}
\inferrule*[right=Sat-\True]{ }{
    \AVALmi \True
}
\and
% \inferrule*[right=Val-\False]{ }{
%     \NAVAL m|I|- \False
% }
% \and 
\inferrule*[right=Sat-Neg]{
    \neg\ \AVALmi P
}{
    \AVALmi \neg P
}
\and
\inferrule*[right=Sat-$\wedge$]{
    \AVALmi P \\
    \AVALmi Q
}{
    \AVALmi P \wedge Q
}
\and \\
\inferrule*[right=Sat-$\vee_L$]{
    \AVALmi P
}{
    \AVALmi P \vee Q
}
\and
\inferrule*[right=Sat-$\vee_R$]{
    \AVALmi Q
}{
    \AVALmi P \vee Q
}
\and
\inferrule*[right=Sat-$\Rightarrow$]{
    \AVAL m|I|- P \Longrightarrow \AVAL m|I|- Q
}{
    \AVALmi P \Longrightarrow Q
}
\and
% \inferrule*[right=Val-$\Rightarrow_L$]{
%     \NAVAL m|I|- P
% }{
%     \AVALmi P \Longrightarrow Q
% }
% \and
% \inferrule*[right=Val-$\Rightarrow_R$]{
%     \AVALmi Q
% }{
%     \AVALmi P \Longrightarrow Q
% }
% \and
\inferrule*[right=Sat-$\forall$]{
    \forall n.\, \AVAL m| I[i \mapsto n] |- P
}{
    \AVALmi \forall i.\, P
}
\and
\inferrule*[right=Sat-$\exists$]{
    %\exists n.\, 
    \AVAL m| I[i \mapsto n] |- P
}{
    \AVALmi \exists i.\, P
}
\and
\inferrule*[right={Sat-$=_A$}]{
    \EvalI(a_1,m,I) = \EvalI(a_2,m,I)
}{
    \AVALmi a_1 = a_2
}
\and
\inferrule*[right=Sat-$<$]{
    \EvalI(a_1,m,I) < \EvalI(a_2,m,I)
}{
    \AVALmi a_1 < a_2
}
\and
\inferrule*[right=Sat-$\leq$]{
    \EvalI(a_1,m,I) \leq \EvalI(a_2,m,I)
}{
    \AVALmi a_1 \leq a_2
}
\and
\inferrule*[right={Sat-$=_B$}]{
    \AVALmi b_1 \iff \AVALmi b_2
}{
    \AVALmi b_1 = b_2
}
\end{mathpar}

Note that in our Coq implementation,
rather than having individual rules
we assert satisfaction
by recursively translating an assertion
into a Coq-level proposition.
%
We then prove the above rules hold
w.r.t.\ this translation.

We prove various properties of assertions and expressions.
Note that when we discuss substitution into boolean expressions,
substitutions only affect free variables ---
they are overridden by quantified variables.

\begin{lemma}
For $b \in \DomBexp$,
$\AVALmi b \iff \Eval(b,m) = \True$.
\end{lemma}

\begin{lemma}
For $n \in \Z$ and $a \in \DomAexpv$,
$\EvalI(a[n/x],m,I) = \EvalI(a,m[x \mapsto n],I)$.
Similarly,
$\EvalI(a[n/i],m,I) = \EvalI(a,m,I[i \mapsto n])$.
\end{lemma}

\begin{lemma}
For $n \in \Z$,
$\AVAL m | I |- P[n/x] \iff \AVAL m[x \mapsto n] | I |- P$.
Similarly,
$\AVAL m | I |- P[n/i] \iff \AVAL m | I[i \mapsto n] |- P$.
\end{lemma}

\begin{corollary}
$\AVALmi \forall i.\, P \iff \forall n.\, \AVALmi P[n/i]$.
Similarly,
$\AVALmi \exists i.\, P \iff \exists n.\, \AVALmi P[n/i]$.
\end{corollary}

\iffalse
We define the \emph{interpretation} of $P,Q \in \DomAssert$
to yield the subset of memories
in which the assertion holds:
$\Interp P \triangleq \SetBuild(m | {\Eval(P,m) = \True})$.
Properties of interpretation include (all proven in Coq):
%
\begin{mathpar}
\begin{array}{rcl}
\Interp{\neg P} & = & \complement{\Interp{P}} \\
\Interp{P \wedge Q} & = & \Interp{P} \cap \Interp{Q} \\
\Interp{P \vee Q} & = & \Interp{P} \cup \Interp{Q}
\end{array}
\and
\begin{array}{rcl}
\Interp{P \Rightarrow Q} & = & \complement{\Interp P} \cup \Interp{Q} \\
\Interp{\ETrue} & = & \Mems  \\
\Interp{\EFalse} & = & \emptyset
\end{array}
\end{mathpar}

A boolean expression is \emph{valid},
denoted $\VAL P$,
if $P$ always yields $\True$,
i.e.\ $\Interp P = \Mems$.
Properties include:
{
%
\newcommand*{\LINE}[3]{%
    \VAL #1 & \text{#3} & #2 \\ %
}
%
\begin{mathpar}
{\BIIMP\inferrule*[right=Val-$\wedge$]{
    \VAL P \\ \VAL Q
}{
    \VAL P \wedge Q
}}
%
\and
%
\inferrule*[right=Val-$\vee_\textsc{L}$]{
    \VAL P
}{
    \VAL P \vee Q
}
%
\and
%
\inferrule*[right=Val-$\vee_\textsc{R}$]{
    \VAL Q
}{
    \VAL P \vee Q
}
%
\and
{\BIIMP\inferrule*[right=Val-$\Rightarrow$]{
    \Interp P \subseteq \Interp Q
}{
    \VAL P \Rightarrow Q
}}
\end{mathpar}
}
\fi



\section{Proof system for unary Hoare logic}

\paragraph{Hoare judgements}
We define partial Hoare triples:
\begin{definition}
    $m$ \emph{satisfies}
    $\Triple c : P => Q$ in $I$,
    denoted $\AVALmi \Triple c : P => Q$,
    iff    
    $$\AVALmi P \implies \forall m'.\
    \Yields(c,m,m') \implies
    %\Cfg(c,m) \sstepstotrans \Cfg(\Skip,m') \implies
    \AVAL m'|I|-Q.$$
%
$\AVALU \Triple c : P => Q$ denotes
that the triple is \emph{valid},
namely $\AVALmi \Triple c : P => Q$
for \emph{all} $m$ and $I$.
\end{definition}

We write $\DER \Triple c : P => Q$ to say that
the triple is \emph{derivable} in Hoare logic.

\iffalse
\begin{displaymath}
\begin{array}{lcl}
    \Triple c : P => Q
    & \triangleq &
    \forall m \in P, m'.\
    \Cfg(c,m) \sstepstotrans \Cfg(\Skip,m') \implies
    m' \in Q
\end{array}
\end{displaymath}
\fi

\begin{figure}[h]
\label{fig:unary-hl-rules}

\paragraph{Structural rules}

\begin{mathpar}
    % Conseq
    \inferrule*[right=HL-Csq]
    { \VAL P \implies P' \\ 
      \DER \Triple c : P' => Q' \\ 
      \VAL Q' \implies Q }
    { \DER \Triple c : P => Q }
    % Case
    \iffalse
    \and 
    \inferrule*[right=HL-Case]
    { % \VAL P \Longrightarrow P' \vee \neg P' \\
      \DER \Triple c : P \wedge P' => Q \\ 
      \DER \Triple c : P \wedge \neg P' => Q }
    { \DER \Triple c : P => Q }
    \and
    %\inferrule*[right=HL-False]
    %{ }{ \DER \Triple c : \EFalse => Q }
    \fi
\end{mathpar}


\paragraph{Symbolic execution rules}


\begin{mathpar}
    %\inferrule*[right=HL-Abort]
    %{ }
    %{ \DER \Triple \Abort : P => Q }
    %\and
    \inferrule*[right=HL-Skip] 
    { }
    { \DER \Triple \Skip : P => P }
    \and
    \inferrule*[right=HL-Assn]
    { }
    { \DER \Triple \Assn x = a : \Subst P[a/x] => P }
    \and
    \inferrule*[right=HL-Seq]
    { \DER \Triple c : P => Q \\ \DER \Triple c' : Q => R }   
    { \DER \Triple \Seq c;c' : P => R }
    \and
    \inferrule*[right=HL-If]
    { \DER \Triple c : P \wedge b => Q \\
        \DER \Triple c' : P \wedge \neg b => Q}
    { \DER \Triple \ITE b then c else c' : P \wedge b => Q }
    \\
    \inferrule*[right=HL-While-T]
    { \DER \Triple \Seq c; \While b do c : P \wedge b => Q }
    { \DER \Triple \While b do c : P \wedge b => Q \wedge \neg b }
    \and
    \inferrule*[right=HL-While-F]
    { }
    { \DER \Triple \While b do c : P \wedge \neg b => P \wedge \neg b }
\end{mathpar}

\caption{All rules for unary Hoare logic.}

\end{figure}


\noindent
These rules are \emph{not} enough to subsume loop invariants, but with cyclic proofs they will be.
These rules are enough to recover a branching \emph{if} rule (using \emph{case}).



\subsection*{Local soundness}
In the case of unary HL,
statements are triples of the form
$\Triple c : P => Q$, and
assertions $P$,
where the latter comprise the liftable statements;
and the rules consist of the structural and symbolic execution rules defined above.


\begin{lemma}
    Each proof rule (structural and symbolic execution) are locally sound.
    %If $\DER \Triple c : P => Q$
    %then $\VAL \Triple c : P => Q$.
\end{lemma}

\noindent
\emph{Proof.}

\subsubsection*{\textsc{HL-Csq}}
We show if $\VAL P \implies P'$ and 
      $\VAL \Triple c : P' => Q'$ and 
      $\VAL Q' \implies Q$
then $\VAL \Triple c : P => Q$.
% We are given that $\forall m \in P : m \in P'$
% and that $\forall m \in Q' : m \in Q$.
The triple we are given unfolds to
\[
    \forall m \in \Interp{P'}, m'.\
    \Cfg(c, m) \sstepstotrans \Cfg(\Skip, m')
    \implies m' \in \Interp{Q'}.
\]
We want to show that
\[
    \forall m \in \Interp P, m'.\
    \Cfg(c, m) \sstepstotrans \Cfg(\Skip, m')
    \implies m' \in \Interp Q.
\]
Note that we are also given $\Interp{P} \subseteq \Interp{P'}$ and $\Interp{Q'} \subseteq \Interp{Q}$,
so the claim immediately follows.

\subsubsection*{\textsc{HL-Case}}
We show if $\ \VAL \Triple c : P \wedge e => Q\ $ and
      $\ \VAL \Triple c : P \wedge \neg e => Q\ $ then
    $\ \VAL \Triple c : P => Q$.
Note that this rule is for Boolean expressions $e$,
so for any memory $m$, we have that $\Eval(e,m) = \True$
or $\Eval(e,m) = \False$.

The two triples unfold to
\[
    \forall m \in \Interp{P \wedge e}, m'.\
    \Cfg(c, m) \sstepstotrans \Cfg(\Skip, m') 
    \implies m' \in Q.
\]
Note

We are given that for any $m$. $m \in P \wedge \Eval(e,m) = \True \implies $
\todo{todo}

\subsubsection*{\textsc{HL-False}}
We show $\VAL \Triple c : \EFalse => P$.
This unfolds to
\[
    \forall m \in \Interp{\EFalse}, m'.\
    \Cfg(c, m) \sstepstotrans \Cfg(\Skip, m') 
    \implies m' \in \Interp{P}.
\]
Note that for any $m$, $m \notin \Interp{\EFalse}$,
so this is vacuously true.

\subsubsection*{\textsc{HL-Skip}}
We show $\VAL \Triple \Skip : P => P$.
This unfolds to 
$$\forall m \in \Interp{P}, m'.\
\Cfg(\Skip,m) \sstepstotrans \Cfg(\Skip,m') \implies
m' \in \Interp{P}.$$
$\Cfg(\Skip,m)$ steps to nothing,
so the only possibility is $m' = m$.
Therefore $m \in \Interp{P} \implies m' = m \in \Interp{P}$.

\subsubsection*{\textsc{HL-Assn}}
We show $\VAL \Triple \Assn x = e : \Subst P[e/x] => P$.
\todo{todo}

\subsubsection*{\textsc{HL-Seq}}
We show if $\VAL \Triple c : P => Q$ and $\VAL \Triple c' : Q => R$
then $\VAL \Triple \Seq c;c' : P => R$.
\todo{todo}
    
\subsubsection*{\textsc{HL-If-T}}
We show if $\VAL \Triple c : P \wedge e => Q$ then
    $\VAL \Triple \ITE e then c else c' : P \wedge e => Q$.
\todo{todo}

\subsubsection*{\textsc{HL-If-F}}
We show if $\VAL \Triple c' : P \wedge \neg e => Q$ then
    $\VAL \Triple \ITE e then c else c' : P \wedge \neg e => Q$.
\todo{todo}

\subsubsection*{\textsc{HL-While-T}}
We show if $\VAL \Triple \Seq c; \While e do c : P \wedge e => Q$ then
    $\VAL \Triple \While e do c : P \wedge e => Q \wedge \neg e$. 
\todo{todo}

\subsubsection*{\textsc{HL-While-F}}
We show~~$\VAL \Triple \While e do c : P \wedge \neg e => P \wedge \neg e$.
This unfolds to
$$\forall m \in P \wedge \neg e, m'.\
\Cfg(\While e do c,m) \sstepstotrans \Cfg(\Skip,m') \implies
m' \in P \wedge \neg e.$$

$m \in P \wedge \neg e$, which means $\Eval(e,m) = \False$
(or $\Eval(e,m) = \bot$, in which case the config cannot step at all and statement holds vacuously),
so $\Cfg(\While e do c,m) \sstepsto \Cfg(\Skip,m)$ by \textsc{SS-While-F}.
Thus $m' = m$, so $m' = m \in P \wedge \neg e$.


\hfill $\qed$

\subsection*{Unary Soundness}
\begin{theorem}
If $\DER \Triple c : P => Q$ then $\VAL \Triple c : P => Q$.
\end{theorem}

\noindent
\emph{Proof.}

\noindent
We derive this result from local soundness.
\todo{todo induction on programs?}
\hfill $\qed$


\subsection{Incompleteness of Acyclic Unary HL}

\begin{lemma}
    For any $P$ and $Q$,
    $\VAL \Triple {\While {\True} do {\Skip} } : P => Q$.
\end{lemma}
\begin{proof}
    By strong induction on the number of execution steps.
\end{proof}

\begin{lemma}
    $\Triple {\While {\True} do {\Skip} } : \True => \False$ is not derivable using acyclic proof graphs.
\end{lemma}
\begin{proof}
Suppose for contradiction that we have a lift-valid, acyclic graph $G$ for which
$$\DERby G {\Triple {\While {\True} do {\Skip} } : \True => \False}$$

We consider two families of statements within $G$: %forms. 
%In all cases, we pres that $P$ and $R$ are any tautologies, and $Q$ is any contradiction.
\begin{enumerate}[label=(\roman*)]
\item $\Triple {\While {\True} do {\Skip} } : P => Q$ where $P$ is a tautology. \\
    The only rules that we can possibly apply to this statement are \textsc{HL-Csq},
    or potentially \textsc{HL-While-T} depending on the shape of $P$ and $Q$.
    %
    \[\begin{prooftree}
        \hypo{\VAL P \Rightarrow P'}
        \hypo{\DER \Triple {\While {\True} do {\Skip} } : P' => Q'}
        \hypo{\VAL Q' \Rightarrow Q}
        \infer3[\textsc{HL-Csq}] 
            {\DER \Triple {\While {\True} do {\Skip} } : P => Q}
    \end{prooftree}\]
%
    Note that because $P$ is a tautology and $\VAL P \Rightarrow P'$,
    it must be $P'$ is a tautology.
    %Similarly, $Q'$ must be a contradiction.
    Thus the middle premise must have form (i).
    
    \[\begin{prooftree}
        \hypo{\DER \Triple {\Seq {\Skip} ; {\While {\True} do {\Skip}} } : P => Q'}
        \infer1[\textsc{HL-While-T}] 
            {\DER \Triple {\While {\True} do {\Skip} } : P => Q }
    \end{prooftree}\]

    Supposing that $P$ is of an appropriate form to apply \textsc{HL-While-T},
    the resulting premise will also have precondition $P$.
    So the premise will have form (ii).
    
    The only other candidate for a rule would be \textsc{HL-While-F}.
    % However, this requires $P$ to have the form $R \wedge \neg \True$.
    However, this requires $P$ to have the form $P' \wedge \neg \True$ for some $P'$,
    which would mean $P$ is not a tautology. %it a contradiction.
    % We assumed $P$ to be a tautology, but $R \wedge \neg \True$ is a contradiction.
    So \textsc{HL-While-F} is not applicable.

\item $\Triple {\Seq \Skip ; {\While {\True} do {\Skip}} } : P => Q$ where $P$ is a tautology. \\
    The only rules we can apply here are \textsc{HL-Csq} and \textsc{HL-Seq}.
%
    \[\begin{prooftree}
        \hypo{\VAL P \Rightarrow P'}
        \hypo{\DER \Triple {\Seq \Skip ; {\While {\True} do {\Skip}} } : P' => Q'}
        \hypo{\VAL Q' \Rightarrow Q}
        \infer3[\textsc{HL-Csq}] 
            {\DER \Triple {\Seq \Skip ; {\While {\True} do {\Skip}} } : P => Q}
    \end{prooftree}\]
%
    By analogous reasoning to that in case (i),
    the middle premise must have form (ii).
    %
    \[\begin{prooftree}
        \hypo{\DER \Triple \Skip : P => R}
        \hypo{\DER \Triple {\While {\True} do {\Skip} } : R => Q}
        \infer2[\textsc{HL-Seq}] 
            {\DER \Triple {\Seq \Skip ; {\While {\True} do {\Skip}} } : P => Q}
    \end{prooftree}\]
%
    Because we assume $G$ to be a valid proof,
    and we have proven acyclic soundness for UHL,
    then $\DER \Triple \Skip : P => R$ appearing in the graph
    must mean that $\VAL \Triple \Skip : P => R$.
    As a result, because $P$ is a tautology, $R$ must then be a tautology.
    Thus the right premise has form (i).
    
%\item $\Triple {\Skip} : P => R$\\
%    If $P = R$, then we can apply \textsc{HL-Skip} to dispatch this statement.
%    We also can always apply \textsc{HL-Csq}.
\end{enumerate}

Observe that any proof of any statement of forms (i) or (ii)
must depend on another statement of one of those forms.
There is also at least one statement in $G$ with one of these forms;
namely 
$\Triple {\While {\True} do {\Skip} } : \True => \False$ is of form (i).
%
We can therefore show inductively that a path of arbitrary length
can be formed by chaining together statements of forms (i) and/or (ii).
This means by Lemma~\ref{lem:cyclic-iff-no-longest-path} that $G$ is cyclic, contradicting our premise.

%$\Triple {\While {\True} do {\Skip} } : \True => \False$ is of form (i).
%Therefore, no acyclic proof of said statement exists.

\end{proof}

\begin{theorem}
    The rules in Fig.~\ref{fig:unary-hl-rules}
    are incomplete for acyclic proof graphs. 
    %Namely, a valid statement $\VAL\Triple c : P => Q$ exists where $\DER \Triple c : P => Q$ is not derivable, even given an oracle for liftable statements.
\end{theorem}
\begin{proof}
By prior lemmata,
$\Triple {\While {\True} do {\Skip} } : \True => \False$
is valid but not derivable.
Thus by definition, the proof rules are incomplete.
\end{proof}



\subsection{Admissible Rules}

\begin{conjecture}
    The following rule is acyclically admissible w.r.t.\
    the rules in \ref{fig:unary-hl-rules}:
    \[ \inferrule*[right=HL-Case]{
        \Triple c : P \wedge e => Q \\
        \Triple c : P \wedge \neg e => Q
    }{
        \Triple c : P => Q
    }
    \]
\end{conjecture}

\section{Proof system for relational Hoare logic}

\paragraph{Relational Hoare judgements}

\begin{displaymath}
\begin{array}{lcl}
    \Quad c_1 ~ c_2 : P => Q
    & \triangleq &
    \forall (m_1,m_2) \in P.\
    \wedge
    \begin{cases}
        \Terminate(c_1,m_1)
        \iff
        \Terminate(c_2,m_2)
        \\
        \forall m'_1, m'_2.\
        \Yields(c_1,m_1,m'_1) \wedge
        \Yields(c_2,m_2,m'_2) \implies
        %\Cfg(c_1,m_1) \sstepstotrans \Cfg(\Skip,m'_1) \wedge
        %\Cfg(c_2,m_2) \sstepstotrans \Cfg(\Skip,m'_2) \\
        %\hspace{0.5in} \implies
        (m'_1,m'_2) \in Q
    \end{cases}
\end{array}
\end{displaymath}

\paragraph{Structural rules}

\begin{mathpar}
    % Conseq
    \inferrule*[right=Conseq]
    { \Quad c_1 ~ c_2 : P' => Q' \\ P \implies P' \\ Q' \implies Q }
    { \Quad c_1 ~ c_2 : P => Q }
    % Case
    \and \inferrule*[right=Case]
    { \Eval(e,m) = \False}
    { \Quad c_1 ~ c_2 : P => Q }
\end{mathpar}


\paragraph{One-sided rules}


\begin{mathpar}
    \inferrule*[right=Assn-L]
    { \Eval(e,m) = \True \\ \Eval(e,m) = \True \\  }
    { \Cfg(\While e do c, m) \sstepsto \Cfg(\Seq c; \While e do c, m)}
    % While false
    \and \inferrule*[right=If-L]
    { \Eval(e,m) = \False}
    { \Cfg(\While e do c, m) \sstepsto \Cfg(\Skip, m)}
\end{mathpar}


\section{Completeness}

\subsection*{Invariants}

We argue that cyclic proofs subsume the one- and two-sided proof rules for loop invariants.


\begin{conjecture}
The following rule holds:
\begin{displaymath}
    \inferrule*[right=Ht-while-inv]
    { \DER \Triple c : P \wedge e => P }
    { \DER \Triple \While e do c : P => P \wedge \neg e }
\end{displaymath}
\end{conjecture}

\noindent
\emph{Proof.}

\noindent
By the definition of Hoare judgements, the proof boils down to proving 
the claim that given 
\[
    \forall m \in \Interp{P \wedge e}, m'.\
    \Cfg(c, m) \sstepstotrans \Cfg(\Skip, m') 
    \implies m' \in \Interp{P},
\]
it is true that 
\[
    \forall m \in \Interp{P}, m'.\
    \Cfg(\While e do c, m) \sstepstotrans \Cfg(\Skip, m')
    \implies m' \in \Interp{P \wedge \neg e}.
\]

We will prove this by induction on the length of the sequence of program-memory pairs that witnesses 
the multi-step reduction in this statement. Note that since $(\While e do c) \neq \Skip$, 
this sequence will always have length at least 1.

Our inductive hypothesis, parametrized by a positive integer $k$ is that, 
for any $m, m'$ where $m \in \Interp{P}$ such that 
$\Cfg(\While e do c, m) \sstepstotrans \Cfg(\Skip, m')$ in $k$ steps, 
we have $m' \in \Interp{P \wedge \neg e}$.
More precisely:
\begin{align*}
    \forall m \in \Interp{P}, m'. 
    \bigl(\exists &\Cfg(c_0, m_0), \Cfg(c_1, m_1), \ldots, \Cfg(c_k, m_k). \\
     &\Cfg(c_0, m_0) = \Cfg(\While e do c, m)
     \wedge \Cfg(c_k, m_k) = \Cfg(\Skip, m') 
     \wedge \forall i \in \{0, \ldots, i-1\}. \Cfg(c_i, m_i) \sstepsto \Cfg(c_{i+1}, m_{i+1}) 
     \bigr) \\
    &\implies m' \in \Interp{P \wedge \neg e}.
\end{align*}

\subsubsection*{Base case} ($k = 1$)

\noindent
If $\Cfg(\While e do c, m) \sstepsto \Cfg(\Skip, m')$, the only small-step operation semantics rule 
that could apply is \textsc{SS-While-F}.
Thereby, it must be that $m' = m$ and $\Eval(e, m) = \False$ (i.e. $m \in \Interp{\neg e}$).
Recall that we also have that $m \in \Interp{P}$ by assumption.
We can then conclude that $m' \in \Interp{P \wedge \neg e}$.

\subsubsection*{Inductive step}\

\noindent
Suppose $m \in \Interp{P}$ and that $\Cfg(\While e do c, m) \sstepstotrans \Cfg(\Skip, m')$ in $k$ steps.
The two small-step operation semantics rule that apply are \textsc{SS-While-T} and \textsc{SS-While-F}.
Since we are necessarily taking a step, it must be the case that $\Eval(e, m) \in \{\True, \False\}$.

In the case that $\Eval(e, m) = \False$, we step to $\Cfg(\Skip, m)$ and cannot step any further.
Hence, we have that $m' = m$, and the base case applies.

% https://www.lri.fr/~marche/MPRI-2-36-1/2012/poly-chap2.pdf
In the case that $\Eval(e, m) = \True$, \todo{recourse to sequential execution lemma; take the \textsc{SS-While-T} step and apply the lemma.}


% we step to some configuration $\Cfg(c_1, m_1)$; 
% which, in turn, steps to $\Cfg(\Skip, m')$ in $k-1$ steps.

% \subsubsection*{\textsc{HL-Csq}}
% We show if $\VAL P \implies P'$ and 
%       $\VAL \Triple c : P' => Q'$ and 
%       $\VAL Q' \implies Q$
% then $\VAL \Triple c : P => Q$.
% % We are given that $\forall m \in P : m \in P'$
% % and that $\forall m \in Q' : m \in Q$.
% The triple we are given unfolds to
% \[
%     \forall m \in \Interp{P'}, m'.\
%     \Cfg(c, m) \sstepstotrans \Cfg(\Skip, m')
%     \implies m' \in \Interp{Q'}.
% \]
% We want to show that
% \[
%     \forall m \in \Interp P, m'.\
%     \Cfg(c, m) \sstepstotrans \Cfg(\Skip, m')
%     \implies m' \in \Interp Q.
% \]
% Note that we are also given $\Interp{P} \subseteq \Interp{P'}$ and $\Interp{Q'} \subseteq \Interp{Q}$,
% so the claim immediately follows.


\newpage
\section{Holding Area}


\paragraph{Symbolic execution rules}


\begin{mathpar}
    %\inferrule*[right=HL-Abort]
    %{ }
    %{ \DER \Triple \Abort : P => Q }
    %\and
    \inferrule*[right=HL-Skip \todo{Is this okay?}] 
    { \VAL P \implies Q }
    { \DER \Triple \Skip : P => Q }
    \and
    \inferrule*[right=HL-Assn]
    { \VAL P \implies Q }
    { \DER \Triple \Assn x = e : \Subst P[e/x] => Q }
    \and
    \inferrule*[right=HL-Seq]
    { \DER \Triple c : P => Q \\ \DER \Triple c' : Q => R }   
    { \DER \Triple \Seq c;c' : P => R }
    \\
    \inferrule*[right=HL-If-T]
    { \VAL P \implies e \\ \DER \Triple c : P => Q }
    { \DER \Triple \ITE e then c else c' : P => Q }
    \and
    \inferrule*[right=HL-If-F]
    { \VAL P \implies \neg e \\ \DER \Triple c' : P => Q }
    { \DER \Triple \ITE e then c else c' : P => Q }
    \\
    \inferrule*[right=HL-While-T]
    { \VAL P \implies e \\ \DER \Triple \Seq c; \While e do c : P => Q }
    { \DER \Triple \While e do c : P => Q }
    \and
    \inferrule*[right=HL-While-F]
    { \VAL P \implies Q \wedge \neg e}
    { \DER \Triple \While e do c : P => Q }
\end{mathpar}



we show if $\VAL P \implies Q$
then $\VAL \Triple \Skip : P => Q$.
The latter unfolds to
$\forall m \in P, m'.\
\Cfg(\Skip,m) \sstepstotrans \Cfg(\Skip,m') \implies
m' \in Q.$ The only option is $m = m'$,
so we need $m \in P \implies m \in Q$,
which is exactly our assumption $\VAL P \implies Q$.




\begin{lemma}
    Determinism: if $\Cfg(c,m) \sstepstotrans \Cfg(\Skip,m')$
    and $\Cfg(c,m) \sstepstotrans \Cfg(\Skip,m'')$,
    then $m'=m''$. \todo{(This may not be needed at all)}
\end{lemma}


\newcommand{\rhl}[4]{
    #1 \sim #2 : #3 \Rightarrow #4
}

\newcommand{\rhlsplit}[4]{
    \begin{gathered}
        #1 \sim #2 \\
        {:}\ #3 \Rightarrow #4
    \end{gathered}
}

\newcommand{\eqr}[1]{
    {=}\hspace{-0.3em}\left\{#1\right\}
}

\newcommand{\dEnv}{\mathsf{Env}}
\newcommand{\dA}{\mathsf{A}}
\newcommand{\dB}{\mathsf{B}}

\newcommand{\pSkip}{\mathsf{skip}}
\newcommand{\one}{ {\{}1{\}} }
\newcommand{\two}{ {\{}2{\}} }

\[\begin{prooftree}
    \infer0{(\eqr{d,x} \wedge d\one = \dEnv) \Rightarrow \eqr{x}}
    \infer1[$\textbf{\textsc{Skip}}$]
        {\rhl{\pSkip}{\pSkip}{\eqr{d,x} \wedge d\one = \dEnv}{\eqr{x}}}
    \infer1[$\textbf{rcondf}_{1, 2}$]
        {\rhl{H_1}{H_2}{\eqr{d,x} \wedge d\one = \dEnv}{\eqr{x}}}
    \infer1[$\textbf{rcondf}_{1}$] 
        {\rhl{U_1}{H_2}{\eqr{d,x} \wedge d\one = \dEnv}{\eqr{x}}}
\end{prooftree}\]

\end{document}

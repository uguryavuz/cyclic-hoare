\documentclass[10pt]{article}
\usepackage[margin=1.2in]{geometry}

\usepackage{xcolor}
\usepackage{amsmath}  % Align math mode
\usepackage{amssymb}  % Symbols
\usepackage{amsthm}   % Numbered thm/def environments
\usepackage{ebproof}  % Inference rules
\usepackage{enumitem} % Enumeration
\usepackage{graphicx} % Graphics
\usepackage{hyperref} % Hyperlinks
\usepackage{mathpazo} % Palatino - was there ever any other choice?
\usepackage{mathtools}
\usepackage{stmaryrd}
\usepackage{setspace} % Line spacing
\usepackage{titlesec}
\usepackage{mathpartir}

\usepackage{mystyle}

\titleformat{\subsubsection}[runin]
  {\normalfont\normalsize\bfseries}{\thesubsubsection}{1em}{}

% Numbered environments
\theoremstyle{definition}
\newtheorem{definition}{Definition}
\newtheorem{example}{Example}
\newtheorem{theorem}{Theorem}
\newtheorem{conjecture}{Conjecture}
\newtheorem{lemma}{Lemma}
\newtheorem*{theorem*}{Theorem}
\newtheorem{corollary}{Corollary}[theorem]

% \title{Cyclic-Proj}
% \author{Ugur Yavuz}
% \date{May 2024}

\begin{document}

\section{Operational semantics of the \textsc{While} language}

\subsection*{Syntax}

\begin{displaymath}
\begin{array}{lrcl}
    & n & \in & \Z \\
    & x & \in  & \domvar \\
    \domvar & v & \bnf & n \ALT \True \ALT \False \\
    \DomAexp & a & \bnf & x \ALT n \ALT a + a \ALT a - a \ALT a \times a \\
    \DomBexp & b & \bnf & \True \ALT \False \ALT 
        \neg b \ALT b \wedge b \ALT b \vee b \ALT b \Rightarrow b \\\
    && \ALT & a = a \ALT b = b \ALT a < a \ALT a \leq a \\
    & e & \bnf & a \ALT b \\ 
    \DomComm & c & \bnf & \Skip \ALT \Assn x = a
        \ALT \Seq c ; c \\
    && | & \ITE b then c else c \\
    && | & \While b do c
\end{array}
\end{displaymath}

\subsection*{Memories and expressions}

Memories (in the family $\Mems$)
are total maps from variables (\domvar) to integers.
%
We define the \emph{evaluation} of an expression $e$
w.r.t.\ a memory $m$, denoted $\Eval(e,m)$,
to yield exactly one value $v \in \Vals$.

{
\newcommand{\LINE}[2]{
    \Eval(#1,m) &\triangleq& #2 \\
}

\begin{mathpar}
\begin{array}{rcl}
\LINE x {m(x)}
\LINE n n
\LINE \True \True
\LINE \False \False
\LINE {\neg b} 
    {\neg \Eval(b,m)}
\LINE {b_1 \wedge b_2} {
    \Eval(b_1,m) \wedge \Eval(b_2,m)}
\LINE {b_1 \vee b_2} {
    \Eval(b_1,m) \vee \Eval(b_2,m)}
\LINE {b_1 \Rightarrow b_2} {
    \Eval(b_1,m) \Rightarrow \Eval(b_2,m)}
\end{array}
\and
\begin{array}{rcl}
\LINE {a_1 = a_2} {
    \Eval(a_1,m) = \Eval(a_2,m)}
\LINE {b_1 = b_2} {
    \Eval(b_1,m) = \Eval(b_2,m)}
\LINE {a_1 < a_2} {
    \Eval(a_1,m) < \Eval(a_2,m)}
\LINE {a_1 \leq a_2} {
    \Eval(a_1,m) \leq \Eval(a_2,m)}
\LINE {a_1 + a_2} {
    \Eval(a_1,m) + \Eval(a_2,m)}
\LINE {a_1 - a_2} {
    \Eval(a_1,m) - \Eval(a_2,m)}
\LINE {a_1 \times a_2} {
    \Eval(a_1,m) \times \Eval(a_2,m)}
\end{array}
\end{mathpar}

}

%Define the support of an expression $e$ as
%$\Support e \triangleq \SetBuild(m | {\Eval(e,m) \neq \bot})$.

%A \emph{boolean expression} is an expression $P$ s.t.
%$\forall m.\ \Eval(P,m) \in \set{\True,\False,\bot}$.
%Call this family of expressions $\BoolExps$.
%
We define the \emph{interpretation} of $P,Q \in \DomBexp$
to yield the subset of memories
in which the expression holds:
$\Interp P \triangleq \SetBuild(m | {\Eval(P,m) = \True})$.
Properties of interpretation include (all proven in Coq):

\begin{displaymath}
\begin{array}{rcl}
\Interp{\neg P} & = & \SetBuild(m | {\Eval(P,m) = \False}) \\
\Interp{P \wedge Q} & = & \Interp{P} \cap \Interp{Q} \\
\Interp{P \vee Q} & = & (\Interp{P} \cup \Interp{Q}) \cap \Support P \cap \Support Q \\
\Interp{P \Rightarrow Q} & = & \Interp{\neg P \vee Q} \\
\Interp{\ETrue} & = & \Mems  \\
\Interp{\EFalse} & = & \emptyset \\
\end{array}
\end{displaymath}

A boolean expression is \emph{valid},
denoted $\VAL P$,
iff $\Support P = \Interp P$.
In other words, whenever $P$ evaluates,
it yields $\True$.
Properties include:
{

\newcommand*{\LINE}[3]{%
    \VAL #1 & \text{#3} & #2 \\
}

%\begin{mathpar}
%    \inferrule{
%        \VAL P \\ \VAL Q
%    }{
%    \VAL P \wedge Q
%    }
%\end{mathpar}

\begin{displaymath}
\begin{array}{rcl}
\LINE {P \wedge Q}{\VAL P \text{ and } \VAL Q}{iff}
\LINE {P \Rightarrow Q}{\Interp P \subseteq \Interp Q}{iff}
\LINE {P \vee Q}{\VAL P \text{ or } \VAL Q}{if}
\end{array}
\end{displaymath}

}


\subsection*{Small-step operational semantics}

\begin{mathpar}
    % Assignment
    \inferrule*[right=SS-Assn]
    { \Eval(a,m)=v }
    { \Cfg(\Assn x = a, m) \sstepsto \Cfg(\Skip, \MemDef m[x->v]) }
    % Sequencing
    \\ \inferrule*[right=SS-Seq]
    { \Cfg(c_0,m) \sstepsto \Cfg(c'_0,m') }
    { \Cfg(\Seq c_0 ; c_1, m) \sstepsto \Cfg(c'_0 ; c_1, m')}
    % Sequencing, skip elimination
    \and \inferrule*[right=SS-Seq-Skip]
    { }
    { \Cfg(\Seq \Skip ; c, m) \sstepsto \Cfg(c, m) }
    % If true
    \and \inferrule*[right=SS-If-T]
    { \Eval(b,m) = \True}
    { \Cfg(\ITE b then c_1 else c_2, m) \sstepsto \Cfg(c_1, m)}
    % If false
    \and \inferrule*[right=SS-If-F]
    { \Eval(b,m) = \False}
    { \Cfg(\ITE b then c_1 else c_2, m) \sstepsto \Cfg(c_2, m)}
    % While true
    \and \inferrule*[right=SS-While-T]
    { \Eval(b,m) = \True}
    { \Cfg(\While b do c, m) \sstepsto \Cfg(\Seq c; \While b do c, m)}
    % While false
    \and \inferrule*[right=SS-While-F]
    { \Eval(b,m) = \False}
    { \Cfg(\While b do c, m) \sstepsto \Cfg(\Skip, m)}
\end{mathpar}

% \[
%     \begin{prooftree}[center=false]
%     \hypo{\Eval(e,m)=v}
%     \infer1{\Cfg(\Assn x = e, m) \sstepsto \Cfg(\Skip, m)}
%     \end{prooftree}
%     \qquad
%     \begin{prooftree}[center=false]
%     \hypo{ \Gamma, A \vdash B }
%     \infer1{ \Gamma \vdash A \to B }
%     \end{prooftree}
% \]


\subsection*{Multi-step reduction}

Let $\sstepstotrans$ denote the reflexive transitive closure of $\sstepsto$. In other words,
$\Cfg(c,m) \sstepstotrans \Cfg(c',m')$ iff
$\Cfg(c_0,m_0),\dots,\Cfg(c_k,m_k)$
exist for $k\geq 0$ s.t.\ $\Cfg(c,m)=\Cfg(c_0,m_0)$
and $\Cfg(c_k,m_k)=\Cfg(c',m')$
and $\forall i \in [0,k).\ \Cfg(c_i,m_i) \sstepstotrans \Cfg(c_{i+1},m_{i+1})$.

\begin{definition}
    $c$ \emph{terminates} on $m$, denoted $\Terminate(c,m)$,
    iff $\exists m'. \Cfg(c,m) \sstepstotrans \Cfg(\Skip,m')$.
    $c$ \emph{diverges} on $m$, denoted $\Diverge(c,m)$,
    iff $\neg \Terminate(c,m)$.
\end{definition}

\begin{lemma}[Sequential execution]
    If $\Cfg(c_1; c_2, m) \sstepstotrans \Cfg(\Skip, m')$, 
    then there exists an intermediate memory $m''$ such that
    $\Cfg(c_1, m) \sstepstotrans \Cfg(\Skip, m'')$
    and $\Cfg(c_2, m'') \sstepstotrans \Cfg(\Skip, m')$.
\end{lemma}
\begin{proof}
    Proof by induction on length of execution. See Coq proof.
\end{proof}

% \subsection{Memories}

\section{Proof system for unary Hoare logic}

\paragraph{Hoare judgements}
We define partial Hoare triples:
\begin{definition}
    Judgment $\Triple c : P => Q$ is \emph{valid}
    for program $c$ and boolean expressions $P$ and $Q$,
    written $\VAL \Triple c : P => Q$, if
    $$\forall m \in \Interp P, m'.\
    \Cfg(c,m) \sstepstotrans \Cfg(\Skip,m') \implies
    m' \in \Interp Q.$$
\end{definition}

We write $\DER \Triple c : P => Q$ to say that
the triple is \emph{derivable} in Hoare logic.

\iffalse
\begin{displaymath}
\begin{array}{lcl}
    \Triple c : P => Q
    & \triangleq &
    \forall m \in P, m'.\
    \Cfg(c,m) \sstepstotrans \Cfg(\Skip,m') \implies
    m' \in Q
\end{array}
\end{displaymath}
\fi


\paragraph{Structural rules}

\begin{mathpar}
    % Conseq
    \inferrule*[right=HL-Csq]
    { \VAL P \implies P' \\ 
      \DER \Triple c : P' => Q' \\ 
      \VAL Q' \implies Q }
    { \DER \Triple c : P => Q }
    % Case
    \and 
    \inferrule*[right=HL-Case]
    { \VAL P \Longrightarrow e \vee \neg e \\
        \DER \Triple c : P \wedge e => Q \\ 
      \DER \Triple c : P \wedge \neg e => Q }
    { \DER \Triple c : P => Q }
    \and
    \inferrule*[right=HL-False]
    { }{ \DER \Triple c : \EFalse => Q }
\end{mathpar}


\paragraph{Symbolic execution rules}


\begin{mathpar}
    %\inferrule*[right=HL-Abort]
    %{ }
    %{ \DER \Triple \Abort : P => Q }
    %\and
    \inferrule*[right=HL-Skip] 
    { }
    { \DER \Triple \Skip : P => P }
    \and
    \inferrule*[right=HL-Assn]
    { }
    { \DER \Triple \Assn x = a : \Subst P[a/x] => P }
    \and
    \inferrule*[right=HL-Seq]
    { \DER \Triple c : P => Q \\ \DER \Triple c' : Q => R }   
    { \DER \Triple \Seq c;c' : P => R }
    \\
    \inferrule*[right=HL-If-T]
    { \DER \Triple c : P \wedge b => Q }
    { \DER \Triple \ITE b then c else c' : P \wedge b => Q }
    \and
    \inferrule*[right=HL-If-F]
    { \DER \Triple c' : P \wedge \neg b => Q }
    { \DER \Triple \ITE b then c else c' : P \wedge \neg b => Q }
    \\
    \inferrule*[right=HL-While-T]
    { \DER \Triple \Seq c; \While b do c : P \wedge b => Q }
    { \DER \Triple \While b do c : P \wedge b => Q \wedge \neg b }
    \and
    \inferrule*[right=HL-While-F]
    { }
    { \DER \Triple \While b do c : P \wedge \neg b => P \wedge \neg b }
\end{mathpar}



\noindent
These rules are \emph{not} enough to subsume loop invariants, but with cyclic proofs they will be.
These rules are enough to recover a branching \emph{if} rule (using \emph{case}).




\subsection*{Local Soundness}


\begin{lemma}
    Each proof rule (structural and symbolic execution) are locally sound.
    %If $\DER \Triple c : P => Q$
    %then $\VAL \Triple c : P => Q$.
\end{lemma}

\noindent
\emph{Proof.}

\subsubsection*{\textsc{HL-Csq}}
We show if $\VAL P \implies P'$ and 
      $\VAL \Triple c : P' => Q'$ and 
      $\VAL Q' \implies Q$
then $\VAL \Triple c : P => Q$.
% We are given that $\forall m \in P : m \in P'$
% and that $\forall m \in Q' : m \in Q$.
The triple we are given unfolds to
\[
    \forall m \in \Interp{P'}, m'.\
    \Cfg(c, m) \sstepstotrans \Cfg(\Skip, m')
    \implies m' \in \Interp{Q'}.
\]
We want to show that
\[
    \forall m \in \Interp P, m'.\
    \Cfg(c, m) \sstepstotrans \Cfg(\Skip, m')
    \implies m' \in \Interp Q.
\]
Note that we are also given $\Interp{P} \subseteq \Interp{P'}$ and $\Interp{Q'} \subseteq \Interp{Q}$,
so the claim immediately follows.

\subsubsection*{\textsc{HL-Case}}
We show if $\ \VAL \Triple c : P \wedge e => Q\ $ and
      $\ \VAL \Triple c : P \wedge \neg e => Q\ $ then
    $\ \VAL \Triple c : P => Q$.
Note that this rule is for Boolean expressions $e$,
so for any memory $m$, we have that $\Eval(e,m) = \True$
or $\Eval(e,m) = \False$.

The two triples unfold to
\[
    \forall m \in \Interp{P \wedge e}, m'.\
    \Cfg(c, m) \sstepstotrans \Cfg(\Skip, m') 
    \implies m' \in Q.
\]
Note

We are given that for any $m$. $m \in P \wedge \Eval(e,m) = \True \implies $
\todo{todo}

\subsubsection*{\textsc{HL-False}}
We show $\VAL \Triple c : \EFalse => P$.
This unfolds to
\[
    \forall m \in \Interp{\EFalse}, m'.\
    \Cfg(c, m) \sstepstotrans \Cfg(\Skip, m') 
    \implies m' \in \Interp{P}.
\]
Note that for any $m$, $m \notin \Interp{\EFalse}$,
so this is vacuously true.

\subsubsection*{\textsc{HL-Skip}}
We show $\VAL \Triple \Skip : P => P$.
This unfolds to 
$$\forall m \in \Interp{P}, m'.\
\Cfg(\Skip,m) \sstepstotrans \Cfg(\Skip,m') \implies
m' \in \Interp{P}.$$
$\Cfg(\Skip,m)$ steps to nothing,
so the only possibility is $m' = m$.
Therefore $m \in \Interp{P} \implies m' = m \in \Interp{P}$.

\subsubsection*{\textsc{HL-Assn}}
We show $\VAL \Triple \Assn x = e : \Subst P[e/x] => P$.
\todo{todo}

\subsubsection*{\textsc{HL-Seq}}
We show if $\VAL \Triple c : P => Q$ and $\VAL \Triple c' : Q => R$
then $\VAL \Triple \Seq c;c' : P => R$.
\todo{todo}
    
\subsubsection*{\textsc{HL-If-T}}
We show if $\VAL \Triple c : P \wedge e => Q$ then
    $\VAL \Triple \ITE e then c else c' : P \wedge e => Q$.
\todo{todo}

\subsubsection*{\textsc{HL-If-F}}
We show if $\VAL \Triple c' : P \wedge \neg e => Q$ then
    $\VAL \Triple \ITE e then c else c' : P \wedge \neg e => Q$.
\todo{todo}

\subsubsection*{\textsc{HL-While-T}}
We show if $\VAL \Triple \Seq c; \While e do c : P \wedge e => Q$ then
    $\VAL \Triple \While e do c : P \wedge e => Q \wedge \neg e$. 
\todo{todo}

\subsubsection*{\textsc{HL-While-F}}
We show~~$\VAL \Triple \While e do c : P \wedge \neg e => P \wedge \neg e$.
This unfolds to
$$\forall m \in P \wedge \neg e, m'.\
\Cfg(\While e do c,m) \sstepstotrans \Cfg(\Skip,m') \implies
m' \in P \wedge \neg e.$$

$m \in P \wedge \neg e$, which means $\Eval(e,m) = \False$
(or $\Eval(e,m) = \bot$, in which case the config cannot step at all and statement holds vacuously),
so $\Cfg(\While e do c,m) \sstepsto \Cfg(\Skip,m)$ by \textsc{SS-While-F}.
Thus $m' = m$, so $m' = m \in P \wedge \neg e$.


\hfill $\qed$

\subsection*{Unary Soundness}
\begin{theorem}
If $\DER \Triple c : P => Q$ then $\VAL \Triple c : P => Q$.
\end{theorem}

\noindent
\emph{Proof.}

\noindent
We derive this result from local soundness.
\todo{todo induction on programs?}
\hfill $\qed$


\newpage
\section{Proof system for relational Hoare logic}

\paragraph{Relational Hoare judgements}

\begin{displaymath}
\begin{array}{lcl}
    \Quad c_1 ~ c_2 : P => Q
    & \triangleq &
    \forall (m_1,m_2) \in P.\
    \wedge
    \begin{cases}
        (\exists m'_1.\ \Cfg(c_1,m_1) \sstepstotrans \Cfg(\Skip,m'_1))
        \iff
        (\exists m'_2.\ \Cfg(c_2,m_2) \sstepstotrans \Cfg(\Skip,m'_2))
        \\
        \forall m'_1, m'_2.\
        \Cfg(c_1,m_1) \sstepstotrans \Cfg(\Skip,m'_1) \wedge
        \Cfg(c_2,m_2) \sstepstotrans \Cfg(\Skip,m'_2) \\
        \hspace{0.5in} \implies
        (m'_1,m'_2) \in Q
    \end{cases}
\end{array}
\end{displaymath}

\paragraph{Structural rules}

\begin{mathpar}
    % Conseq
    \inferrule*[right=Conseq]
    { \Quad c_1 ~ c_2 : P' => Q' \\ P \implies P' \\ Q' \implies Q }
    { \Quad c_1 ~ c_2 : P => Q }
    % Case
    \and \inferrule*[right=Case]
    { \Eval(e,m) = \False}
    { \Quad c_1 ~ c_2 : P => Q }
\end{mathpar}


\paragraph{One-sided rules}


\begin{mathpar}
    \inferrule*[right=Assn-L]
    { \Eval(e,m) = \True \\ \Eval(e,m) = \True \\  }
    { \Cfg(\While e do c, m) \sstepsto \Cfg(\Seq c; \While e do c, m)}
    % While false
    \and \inferrule*[right=If-L]
    { \Eval(e,m) = \False}
    { \Cfg(\While e do c, m) \sstepsto \Cfg(\Skip, m)}
\end{mathpar}


\section{Proof Graphs}

\begin{definition}
    A proof graph is a finite \todo{todo}

    We assume proof graphs to be minimal, i.e.\ any nodes with equivalent values
    are identical nodes in the graph.
\end{definition}

\begin{definition}
    A proof graph is \emph{locally sound} iff every node is locally sound,
    and every leaf $\VAL P \implies Q$ holds.
    A node is locally sound iff \todo{todo}
\end{definition}

\begin{conjecture}
    If $\DER \Triple c : P => Q$ is on a node of a
    acyclic locally sound proof graph,
    then $\VAL \Triple c : P => Q$.
\end{conjecture}

\begin{definition}
    A simple cycle in a proof graph is a sequence of mutually distinct nodes $n_1,\dots,n_k$ for $k > 0$
    s.t.\ $n_i$ points to $n_{i+1}$ for all $i$, and $n_k$ points to $n_1$.
    Recall the assumption that proof graphs contain no distinct duplicate nodes.
\end{definition}

\begin{conjecture}
    Every cyclic proof graph contains finitely many simple cycles.
\end{conjecture}

%\begin{conjecture}
%    Every infinite path in a cyclic proof graph
%    consists of infinite traversals of one or more simple cycles.
%\end{conjecture}

\begin{conjecture}
    If every simple cycle in a proof graph
    contains an instance of \textsc{HL-While-T},
    then every infinite path in the graph
    contains infinitely many instances of \textsc{HL-While-T}.
\end{conjecture}

\begin{conjecture}
    Consider a locally sound proof graph,
    for which every simple cycle contains a node
    with \textsc{HL-While-T}.
    If $\DER \Triple c : P => Q$ is on a node of this graph,
    then $\VAL \Triple c : P => Q$.
\end{conjecture}



\section{Completeness}

\subsection*{Invariants}

We argue that cyclic proofs subsume the one- and two-sided proof rules for loop invariants.


\begin{conjecture}
The following rule holds:
\begin{displaymath}
    \inferrule*[right=Ht-while-inv]
    { \DER \Triple c : P \wedge e => P }
    { \DER \Triple \While e do c : P => P \wedge \neg e }
\end{displaymath}
\end{conjecture}

\noindent
\emph{Proof.}

\noindent
By the definition of Hoare judgements, the proof boils down to proving 
the claim that given 
\[
    \forall m \in \Interp{P \wedge e}, m'.\
    \Cfg(c, m) \sstepstotrans \Cfg(\Skip, m') 
    \implies m' \in \Interp{P},
\]
it is true that 
\[
    \forall m \in \Interp{P}, m'.\
    \Cfg(\While e do c, m) \sstepstotrans \Cfg(\Skip, m')
    \implies m' \in \Interp{P \wedge \neg e}.
\]

We will prove this by induction on the length of the sequence of program-memory pairs that witnesses 
the multi-step reduction in this statement. Note that since $(\While e do c) \neq \Skip$, 
this sequence will always have length at least 1.

Our inductive hypothesis, parametrized by a positive integer $k$ is that, 
for any $m, m'$ where $m \in \Interp{P}$ such that 
$\Cfg(\While e do c, m) \sstepstotrans \Cfg(\Skip, m')$ in $k$ steps, 
we have $m' \in \Interp{P \wedge \neg e}$.
More precisely:
\begin{align*}
    \forall m \in \Interp{P}, m'. 
    \bigl(\exists &\Cfg(c_0, m_0), \Cfg(c_1, m_1), \ldots, \Cfg(c_k, m_k). \\
     &\Cfg(c_0, m_0) = \Cfg(\While e do c, m)
     \wedge \Cfg(c_k, m_k) = \Cfg(\Skip, m') 
     \wedge \forall i \in \{0, \ldots, i-1\}. \Cfg(c_i, m_i) \sstepsto \Cfg(c_{i+1}, m_{i+1}) 
     \bigr) \\
    &\implies m' \in \Interp{P \wedge \neg e}.
\end{align*}

\subsubsection*{Base case} ($k = 1$)

\noindent
If $\Cfg(\While e do c, m) \sstepsto \Cfg(\Skip, m')$, the only small-step operation semantics rule 
that could apply is \textsc{SS-While-F}.
Thereby, it must be that $m' = m$ and $\Eval(e, m) = \False$ (i.e. $m \in \Interp{\neg e}$).
Recall that we also have that $m \in \Interp{P}$ by assumption.
We can then conclude that $m' \in \Interp{P \wedge \neg e}$.

\subsubsection*{Inductive step}\

\noindent
Suppose $m \in \Interp{P}$ and that $\Cfg(\While e do c, m) \sstepstotrans \Cfg(\Skip, m')$ in $k$ steps.
The two small-step operation semantics rule that apply are \textsc{SS-While-T} and \textsc{SS-While-F}.
Since we are necessarily taking a step, it must be the case that $\Eval(e, m) \in \{\True, \False\}$.

In the case that $\Eval(e, m) = \False$, we step to $\Cfg(\Skip, m)$ and cannot step any further.
Hence, we have that $m' = m$, and the base case applies.

% https://www.lri.fr/~marche/MPRI-2-36-1/2012/poly-chap2.pdf
In the case that $\Eval(e, m) = \True$, \todo{recourse to sequential execution lemma; take the \textsc{SS-While-T} step and apply the lemma.}


% we step to some configuration $\Cfg(c_1, m_1)$; 
% which, in turn, steps to $\Cfg(\Skip, m')$ in $k-1$ steps.

% \subsubsection*{\textsc{HL-Csq}}
% We show if $\VAL P \implies P'$ and 
%       $\VAL \Triple c : P' => Q'$ and 
%       $\VAL Q' \implies Q$
% then $\VAL \Triple c : P => Q$.
% % We are given that $\forall m \in P : m \in P'$
% % and that $\forall m \in Q' : m \in Q$.
% The triple we are given unfolds to
% \[
%     \forall m \in \Interp{P'}, m'.\
%     \Cfg(c, m) \sstepstotrans \Cfg(\Skip, m')
%     \implies m' \in \Interp{Q'}.
% \]
% We want to show that
% \[
%     \forall m \in \Interp P, m'.\
%     \Cfg(c, m) \sstepstotrans \Cfg(\Skip, m')
%     \implies m' \in \Interp Q.
% \]
% Note that we are also given $\Interp{P} \subseteq \Interp{P'}$ and $\Interp{Q'} \subseteq \Interp{Q}$,
% so the claim immediately follows.


\newpage
\section{Holding Area}


\paragraph{Symbolic execution rules}


\begin{mathpar}
    %\inferrule*[right=HL-Abort]
    %{ }
    %{ \DER \Triple \Abort : P => Q }
    %\and
    \inferrule*[right=HL-Skip \todo{Is this okay?}] 
    { \VAL P \implies Q }
    { \DER \Triple \Skip : P => Q }
    \and
    \inferrule*[right=HL-Assn]
    { \VAL P \implies Q }
    { \DER \Triple \Assn x = e : \Subst P[e/x] => Q }
    \and
    \inferrule*[right=HL-Seq]
    { \DER \Triple c : P => Q \\ \DER \Triple c' : Q => R }   
    { \DER \Triple \Seq c;c' : P => R }
    \\
    \inferrule*[right=HL-If-T]
    { \VAL P \implies e \\ \DER \Triple c : P => Q }
    { \DER \Triple \ITE e then c else c' : P => Q }
    \and
    \inferrule*[right=HL-If-F]
    { \VAL P \implies \neg e \\ \DER \Triple c' : P => Q }
    { \DER \Triple \ITE e then c else c' : P => Q }
    \\
    \inferrule*[right=HL-While-T]
    { \VAL P \implies e \\ \DER \Triple \Seq c; \While e do c : P => Q }
    { \DER \Triple \While e do c : P => Q }
    \and
    \inferrule*[right=HL-While-F]
    { \VAL P \implies Q \wedge \neg e}
    { \DER \Triple \While e do c : P => Q }
\end{mathpar}



we show if $\VAL P \implies Q$
then $\VAL \Triple \Skip : P => Q$.
The latter unfolds to
$\forall m \in P, m'.\
\Cfg(\Skip,m) \sstepstotrans \Cfg(\Skip,m') \implies
m' \in Q.$ The only option is $m = m'$,
so we need $m \in P \implies m \in Q$,
which is exactly our assumption $\VAL P \implies Q$.




\begin{lemma}
    Determinism: if $\Cfg(c,m) \sstepstotrans \Cfg(\Skip,m')$
    and $\Cfg(c,m) \sstepstotrans \Cfg(\Skip,m'')$,
    then $m'=m''$. \todo{(This may not be needed at all)}
\end{lemma}



\end{document}

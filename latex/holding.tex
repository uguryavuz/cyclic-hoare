\section{Holding Area}


\paragraph{Symbolic execution rules}


\begin{mathpar}
    %\inferrule*[right=HL-Abort]
    %{ }
    %{ \DER \Triple \Abort : P => Q }
    %\and
    \inferrule*[right=HL-Skip \todo{Is this okay?}] 
    { \VAL P \implies Q }
    { \DER \Triple \Skip : P => Q }
    \and
    \inferrule*[right=HL-Assn]
    { \VAL P \implies Q }
    { \DER \Triple \Assn x = e : \Subst P[e/x] => Q }
    \and
    \inferrule*[right=HL-Seq]
    { \DER \Triple c : P => Q \\ \DER \Triple c' : Q => R }   
    { \DER \Triple \Seq c;c' : P => R }
    \\
    \inferrule*[right=HL-If-T]
    { \VAL P \implies e \\ \DER \Triple c : P => Q }
    { \DER \Triple \ITE e then c else c' : P => Q }
    \and
    \inferrule*[right=HL-If-F]
    { \VAL P \implies \neg e \\ \DER \Triple c' : P => Q }
    { \DER \Triple \ITE e then c else c' : P => Q }
    \\
    \inferrule*[right=HL-While-T]
    { \VAL P \implies e \\ \DER \Triple \Seq c; \While e do c : P => Q }
    { \DER \Triple \While e do c : P => Q }
    \and
    \inferrule*[right=HL-While-F]
    { \VAL P \implies Q \wedge \neg e}
    { \DER \Triple \While e do c : P => Q }
\end{mathpar}



we show if $\VAL P \implies Q$
then $\VAL \Triple \Skip : P => Q$.
The latter unfolds to
$\forall m \in P, m'.\
\Cfg(\Skip,m) \sstepstotrans \Cfg(\Skip,m') \implies
m' \in Q.$ The only option is $m = m'$,
so we need $m \in P \implies m \in Q$,
which is exactly our assumption $\VAL P \implies Q$.




\begin{lemma}
    Determinism: if $\Cfg(c,m) \sstepstotrans \Cfg(\Skip,m')$
    and $\Cfg(c,m) \sstepstotrans \Cfg(\Skip,m'')$,
    then $m'=m''$. \todo{(This may not be needed at all)}
\end{lemma}

\subsection{Incompleteness of Acyclic Unary HL}

\begin{lemma}
    For any $P$ and $Q$,
    $\VAL \Triple {\While {\True} do {\Skip} } : P => Q$.
\end{lemma}
\begin{proof}
    By strong induction on the number of execution steps.
\end{proof}

\begin{lemma}
    $\Triple {\While {\True} do {\Skip} } : \True => \False$ is not derivable using acyclic proof graphs.
\end{lemma}
\begin{proof}
We consider a family of statements. 
In all cases, we suppose that $P$ and $R$ are any tautologies, and $Q$ is any contradiction.
\begin{enumerate}[label=(\roman*)]
\item $\Triple {\While {\True} do {\Skip} } : P => Q$ \\
    The only rules that we can possibly apply to this statement are \textsc{HL-Csq},
    or potentially \textsc{HL-While-T}. The latter is addressed in (ii).
    %
    \[\begin{prooftree}
        \hypo{\VAL P \Rightarrow P'}
        \hypo{\DER \Triple {\While {\True} do {\Skip} } : P' => Q'}
        \hypo{\VAL Q' \Rightarrow Q}
        \infer3[\textsc{HL-Csq}] 
            {\DER \Triple {\While {\True} do {\Skip} } : P => Q}
    \end{prooftree}\]
%
    Note that because $P$ is a tautology and $\VAL P \Rightarrow P'$,
    it must be $P'$ is a tautology.
    Similarly, $Q'$ must be a contradiction.
    Thus the middle premise must have form (i).

    The only other candidate for a rule would be \textsc{HL-While-F}.
    % However, this requires $P$ to have the form $R \wedge \neg \True$.
    However, this requires $P$ to end with $\wedge \neg \True$,
    which would make it a contradiction.
    % We assumed $P$ to be a tautology, but $R \wedge \neg \True$ is a contradiction.
    So \textsc{HL-While-F} is never applicable.

\item $\Triple {\While {\True} do {\Skip} } : P \wedge \True => Q \wedge \neg \True$ \\
    The only rules that we can possibly apply to this statement are \textsc{HL-Csq},
    or \textsc{HL-While-T}. As our pre- and post-condition here are respectively a tautology and contradiction, \textsc{HL-Csq} is a special case of (i).
    %
    \[\begin{prooftree}
        \hypo{\DER \Triple {\Seq {\Skip} ; {\While {\True} do {\Skip}} } : P \wedge \True => Q}
        \infer1[\textsc{HL-While-T}] 
            {\DER \Triple {\While {\True} do {\Skip} } : P \wedge \True => Q \wedge \neg \True}
    \end{prooftree}\]
%
    Note that the premise has the form (iii).
    
\item $\Triple {\Seq \Skip ; {\While {\True} do {\Skip}} } : P => Q$ \\
    The only rules we can apply here are \textsc{HL-Csq} and \textsc{HL-Seq}.
%
    \[\begin{prooftree}
        \hypo{\VAL P \Rightarrow P'}
        \hypo{\DER \Triple {\Seq \Skip ; {\While {\True} do {\Skip}} } : P' => Q'}
        \hypo{\VAL Q' \Rightarrow Q}
        \infer3[\textsc{HL-Csq}] 
            {\DER \Triple {\Seq \Skip ; {\While {\True} do {\Skip}} } : P => Q}
    \end{prooftree}\]
%
    By analogous reasoning to that in case (i),
    the middle premise must have form (iii).
    %
    \[\begin{prooftree}
        \hypo{\DER \Triple \Skip : P => R}
        \hypo{\DER \Triple {\While {\True} do {\Skip} } : R => Q}
        \infer2[\textsc{HL-Seq}] 
            {\DER \Triple {\Seq \Skip ; {\While {\True} do {\Skip}} } : P => Q}
    \end{prooftree}\]
%
    Consider that because $P$ is a tautology, $R$ must be a tautology; 
    otherwise the left premise could not possibly be derivable.
    Thus the right premise has form (i).
    
%\item $\Triple {\Skip} : P => R$\\
%    If $P = R$, then we can apply \textsc{HL-Skip} to dispatch this statement.
%    We also can always apply \textsc{HL-Csq}.
\end{enumerate}

Observe that any proof of any statement of forms (i), (ii), or (iii)
must depend on another statement of one of those forms.
Thus the proof graph (which is required to be finite) must be cyclic.

$\Triple {\While {\True} do {\Skip} } : \True => \False$ is of form (i),
so it must be that any proof of said statement is cyclic.

\end{proof}



\section{Proof Graphs}

\begin{definition}
    A proof graph is a finite \todo{todo}

    We assume proof graphs to be minimal, i.e.\ any nodes with equivalent values
    are identical nodes in the graph.
\end{definition}

\begin{definition}
    A proof graph is \emph{locally sound} iff every node is locally sound,
    and every leaf $\VAL P \implies Q$ holds.
    A node is locally sound iff \todo{todo}
\end{definition}

\begin{conjecture}
    If $\DER \Triple c : P => Q$ is on a node of a
    acyclic locally sound proof graph,
    then $\VAL \Triple c : P => Q$.
\end{conjecture}

\begin{definition}
    A simple cycle in a proof graph is a sequence of mutually distinct nodes $n_1,\dots,n_k$ for $k > 0$
    s.t.\ $n_i$ points to $n_{i+1}$ for all $i$, and $n_k$ points to $n_1$.
    Recall the assumption that proof graphs contain no distinct duplicate nodes.
\end{definition}

\begin{conjecture}
    Every cyclic proof graph contains finitely many simple cycles.
\end{conjecture}

%\begin{conjecture}
%    Every infinite path in a cyclic proof graph
%    consists of infinite traversals of one or more simple cycles.
%\end{conjecture}

\begin{conjecture}
    If every simple cycle in a proof graph
    contains an instance of \textsc{HL-While-T},
    then every infinite path in the graph
    contains infinitely many instances of \textsc{HL-While-T}.
\end{conjecture}

\begin{conjecture}
    Consider a locally sound proof graph,
    for which every simple cycle contains a node
    with \textsc{HL-While-T}.
    If $\DER \Triple c : P => Q$ is on a node of this graph,
    then $\VAL \Triple c : P => Q$.
\end{conjecture}




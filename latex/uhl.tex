
\section{Proof system for unary Hoare logic}

\paragraph{Hoare judgements}
We define partial Hoare triples:
\begin{definition}
    $m$ \emph{satisfies}
    $\Triple c : P => Q$ in $I$,
    denoted $\AVALmi \Triple c : P => Q$,
    iff    
    $$\AVALmi P \implies \forall m'.\
    \Yields(c,m,m') \implies
    %\Cfg(c,m) \sstepstotrans \Cfg(\Skip,m') \implies
    \AVAL m'|I|-Q.$$
%
$\AVALU \Triple c : P => Q$ denotes
that the triple is \emph{valid},
namely $\AVALmi \Triple c : P => Q$
for \emph{all} $m$ and $I$.
\end{definition}

We write $\DER \Triple c : P => Q$ to say that
the triple is \emph{derivable} in Hoare logic.

\iffalse
\begin{displaymath}
\begin{array}{lcl}
    \Triple c : P => Q
    & \triangleq &
    \forall m \in P, m'.\
    \Cfg(c,m) \sstepstotrans \Cfg(\Skip,m') \implies
    m' \in Q
\end{array}
\end{displaymath}
\fi

\begin{figure}[h]
\label{fig:unary-hl-rules}

\paragraph{Structural rules}

\begin{mathpar}
    % Conseq
    \inferrule*[right=HL-Csq]
    { \VAL P \implies P' \\ 
      \DER \Triple c : P' => Q' \\ 
      \VAL Q' \implies Q }
    { \DER \Triple c : P => Q }
    % Case
    \iffalse
    \and 
    \inferrule*[right=HL-Case]
    { % \VAL P \Longrightarrow P' \vee \neg P' \\
      \DER \Triple c : P \wedge P' => Q \\ 
      \DER \Triple c : P \wedge \neg P' => Q }
    { \DER \Triple c : P => Q }
    \and
    %\inferrule*[right=HL-False]
    %{ }{ \DER \Triple c : \EFalse => Q }
    \fi
\end{mathpar}


\paragraph{Symbolic execution rules}


\begin{mathpar}
    %\inferrule*[right=HL-Abort]
    %{ }
    %{ \DER \Triple \Abort : P => Q }
    %\and
    \inferrule*[right=HL-Skip] 
    { }
    { \DER \Triple \Skip : P => P }
    \and
    \inferrule*[right=HL-Assn]
    { }
    { \DER \Triple \Assn x = a : \Subst P[a/x] => P }
    \and
    \inferrule*[right=HL-Seq]
    { \DER \Triple c : P => Q \\ \DER \Triple c' : Q => R }   
    { \DER \Triple \Seq c;c' : P => R }
    \and
    \inferrule*[right=HL-If]
    { \DER \Triple c : P \wedge b => Q \\
        \DER \Triple c' : P \wedge \neg b => Q}
    { \DER \Triple \ITE b then c else c' : P \wedge b => Q }
    \\
    \inferrule*[right=HL-While-T]
    { \DER \Triple \Seq c; \While b do c : P \wedge b => Q }
    { \DER \Triple \While b do c : P \wedge b => Q \wedge \neg b }
    \and
    \inferrule*[right=HL-While-F]
    { }
    { \DER \Triple \While b do c : P \wedge \neg b => P \wedge \neg b }
\end{mathpar}

\caption{All rules for unary Hoare logic.}

\end{figure}


\noindent
These rules are \emph{not} enough to subsume loop invariants, but with cyclic proofs they will be.
These rules are enough to recover a branching \emph{if} rule (using \emph{case}).



\subsection*{Local soundness}
In the case of unary HL,
statements are triples of the form
$\Triple c : P => Q$, and
assertions $P$,
where the latter comprise the liftable statements;
and the rules consist of the structural and symbolic execution rules defined above.


\begin{lemma}
    Each proof rule (structural and symbolic execution) are locally sound.
    %If $\DER \Triple c : P => Q$
    %then $\VAL \Triple c : P => Q$.
\end{lemma}

\noindent
\emph{Proof.}

\subsubsection*{\textsc{HL-Csq}}
We show if $\VAL P \implies P'$ and 
      $\VAL \Triple c : P' => Q'$ and 
      $\VAL Q' \implies Q$
then $\VAL \Triple c : P => Q$.
% We are given that $\forall m \in P : m \in P'$
% and that $\forall m \in Q' : m \in Q$.
The triple we are given unfolds to
\[
    \forall m \in \Interp{P'}, m'.\
    \Cfg(c, m) \sstepstotrans \Cfg(\Skip, m')
    \implies m' \in \Interp{Q'}.
\]
We want to show that
\[
    \forall m \in \Interp P, m'.\
    \Cfg(c, m) \sstepstotrans \Cfg(\Skip, m')
    \implies m' \in \Interp Q.
\]
Note that we are also given $\Interp{P} \subseteq \Interp{P'}$ and $\Interp{Q'} \subseteq \Interp{Q}$,
so the claim immediately follows.

\subsubsection*{\textsc{HL-Case}}
We show if $\ \VAL \Triple c : P \wedge e => Q\ $ and
      $\ \VAL \Triple c : P \wedge \neg e => Q\ $ then
    $\ \VAL \Triple c : P => Q$.
Note that this rule is for Boolean expressions $e$,
so for any memory $m$, we have that $\Eval(e,m) = \True$
or $\Eval(e,m) = \False$.

The two triples unfold to
\[
    \forall m \in \Interp{P \wedge e}, m'.\
    \Cfg(c, m) \sstepstotrans \Cfg(\Skip, m') 
    \implies m' \in Q.
\]
Note

We are given that for any $m$. $m \in P \wedge \Eval(e,m) = \True \implies $
\todo{todo}

\subsubsection*{\textsc{HL-False}}
We show $\VAL \Triple c : \EFalse => P$.
This unfolds to
\[
    \forall m \in \Interp{\EFalse}, m'.\
    \Cfg(c, m) \sstepstotrans \Cfg(\Skip, m') 
    \implies m' \in \Interp{P}.
\]
Note that for any $m$, $m \notin \Interp{\EFalse}$,
so this is vacuously true.

\subsubsection*{\textsc{HL-Skip}}
We show $\VAL \Triple \Skip : P => P$.
This unfolds to 
$$\forall m \in \Interp{P}, m'.\
\Cfg(\Skip,m) \sstepstotrans \Cfg(\Skip,m') \implies
m' \in \Interp{P}.$$
$\Cfg(\Skip,m)$ steps to nothing,
so the only possibility is $m' = m$.
Therefore $m \in \Interp{P} \implies m' = m \in \Interp{P}$.

\subsubsection*{\textsc{HL-Assn}}
We show $\VAL \Triple \Assn x = e : \Subst P[e/x] => P$.
\todo{todo}

\subsubsection*{\textsc{HL-Seq}}
We show if $\VAL \Triple c : P => Q$ and $\VAL \Triple c' : Q => R$
then $\VAL \Triple \Seq c;c' : P => R$.
\todo{todo}
    
\subsubsection*{\textsc{HL-If-T}}
We show if $\VAL \Triple c : P \wedge e => Q$ then
    $\VAL \Triple \ITE e then c else c' : P \wedge e => Q$.
\todo{todo}

\subsubsection*{\textsc{HL-If-F}}
We show if $\VAL \Triple c' : P \wedge \neg e => Q$ then
    $\VAL \Triple \ITE e then c else c' : P \wedge \neg e => Q$.
\todo{todo}

\subsubsection*{\textsc{HL-While-T}}
We show if $\VAL \Triple \Seq c; \While e do c : P \wedge e => Q$ then
    $\VAL \Triple \While e do c : P \wedge e => Q \wedge \neg e$. 
\todo{todo}

\subsubsection*{\textsc{HL-While-F}}
We show~~$\VAL \Triple \While e do c : P \wedge \neg e => P \wedge \neg e$.
This unfolds to
$$\forall m \in P \wedge \neg e, m'.\
\Cfg(\While e do c,m) \sstepstotrans \Cfg(\Skip,m') \implies
m' \in P \wedge \neg e.$$

$m \in P \wedge \neg e$, which means $\Eval(e,m) = \False$
(or $\Eval(e,m) = \bot$, in which case the config cannot step at all and statement holds vacuously),
so $\Cfg(\While e do c,m) \sstepsto \Cfg(\Skip,m)$ by \textsc{SS-While-F}.
Thus $m' = m$, so $m' = m \in P \wedge \neg e$.


\hfill $\qed$

\subsection*{Unary Soundness}
\begin{theorem}
If $\DER \Triple c : P => Q$ then $\VAL \Triple c : P => Q$.
\end{theorem}

\noindent
\emph{Proof.}

\noindent
We derive this result from local soundness.
\todo{todo induction on programs?}
\hfill $\qed$


\subsection{Incompleteness of Acyclic Unary HL}

\begin{lemma}
    For any $P$ and $Q$,
    $\VAL \Triple {\While {\True} do {\Skip} } : P => Q$.
\end{lemma}
\begin{proof}
    By strong induction on the number of execution steps.
\end{proof}

\begin{lemma}
    $\Triple {\While {\True} do {\Skip} } : \True => \False$ is not derivable using acyclic proof graphs.
\end{lemma}
\begin{proof}
Suppose for contradiction that we have a lift-valid, acyclic graph $G$ for which
$$\DERby G {\Triple {\While {\True} do {\Skip} } : \True => \False}$$

We consider two families of statements within $G$: %forms. 
%In all cases, we pres that $P$ and $R$ are any tautologies, and $Q$ is any contradiction.
\begin{enumerate}[label=(\roman*)]
\item $\Triple {\While {\True} do {\Skip} } : P => Q$ where $P$ is a tautology. \\
    The only rules that we can possibly apply to this statement are \textsc{HL-Csq},
    or potentially \textsc{HL-While-T} depending on the shape of $P$ and $Q$.
    %
    \[\begin{prooftree}
        \hypo{\VAL P \Rightarrow P'}
        \hypo{\DER \Triple {\While {\True} do {\Skip} } : P' => Q'}
        \hypo{\VAL Q' \Rightarrow Q}
        \infer3[\textsc{HL-Csq}] 
            {\DER \Triple {\While {\True} do {\Skip} } : P => Q}
    \end{prooftree}\]
%
    Note that because $P$ is a tautology and $\VAL P \Rightarrow P'$,
    it must be $P'$ is a tautology.
    %Similarly, $Q'$ must be a contradiction.
    Thus the middle premise must have form (i).
    
    \[\begin{prooftree}
        \hypo{\DER \Triple {\Seq {\Skip} ; {\While {\True} do {\Skip}} } : P => Q'}
        \infer1[\textsc{HL-While-T}] 
            {\DER \Triple {\While {\True} do {\Skip} } : P => Q }
    \end{prooftree}\]

    Supposing that $P$ is of an appropriate form to apply \textsc{HL-While-T},
    the resulting premise will also have precondition $P$.
    So the premise will have form (ii).
    
    The only other candidate for a rule would be \textsc{HL-While-F}.
    % However, this requires $P$ to have the form $R \wedge \neg \True$.
    However, this requires $P$ to have the form $P' \wedge \neg \True$ for some $P'$,
    which would mean $P$ is not a tautology. %it a contradiction.
    % We assumed $P$ to be a tautology, but $R \wedge \neg \True$ is a contradiction.
    So \textsc{HL-While-F} is not applicable.

\item $\Triple {\Seq \Skip ; {\While {\True} do {\Skip}} } : P => Q$ where $P$ is a tautology. \\
    The only rules we can apply here are \textsc{HL-Csq} and \textsc{HL-Seq}.
%
    \[\begin{prooftree}
        \hypo{\VAL P \Rightarrow P'}
        \hypo{\DER \Triple {\Seq \Skip ; {\While {\True} do {\Skip}} } : P' => Q'}
        \hypo{\VAL Q' \Rightarrow Q}
        \infer3[\textsc{HL-Csq}] 
            {\DER \Triple {\Seq \Skip ; {\While {\True} do {\Skip}} } : P => Q}
    \end{prooftree}\]
%
    By analogous reasoning to that in case (i),
    the middle premise must have form (ii).
    %
    \[\begin{prooftree}
        \hypo{\DER \Triple \Skip : P => R}
        \hypo{\DER \Triple {\While {\True} do {\Skip} } : R => Q}
        \infer2[\textsc{HL-Seq}] 
            {\DER \Triple {\Seq \Skip ; {\While {\True} do {\Skip}} } : P => Q}
    \end{prooftree}\]
%
    Because we assume $G$ to be a valid proof,
    and we have proven acyclic soundness for UHL,
    then $\DER \Triple \Skip : P => R$ appearing in the graph
    must mean that $\VAL \Triple \Skip : P => R$.
    As a result, because $P$ is a tautology, $R$ must then be a tautology.
    Thus the right premise has form (i).
    
%\item $\Triple {\Skip} : P => R$\\
%    If $P = R$, then we can apply \textsc{HL-Skip} to dispatch this statement.
%    We also can always apply \textsc{HL-Csq}.
\end{enumerate}

Observe that any proof of any statement of forms (i) or (ii)
must depend on another statement of one of those forms.
There is also at least one statement in $G$ with one of these forms;
namely 
$\Triple {\While {\True} do {\Skip} } : \True => \False$ is of form (i).
%
We can therefore show inductively that a path of arbitrary length
can be formed by chaining together statements of forms (i) and/or (ii).
This means by Lemma~\ref{lem:cyclic-iff-no-longest-path} that $G$ is cyclic, contradicting our premise.

%$\Triple {\While {\True} do {\Skip} } : \True => \False$ is of form (i).
%Therefore, no acyclic proof of said statement exists.

\end{proof}

\begin{theorem}
    The rules in Fig.~\ref{fig:unary-hl-rules}
    are incomplete for acyclic proof graphs. 
    %Namely, a valid statement $\VAL\Triple c : P => Q$ exists where $\DER \Triple c : P => Q$ is not derivable, even given an oracle for liftable statements.
\end{theorem}
\begin{proof}
By prior lemmata,
$\Triple {\While {\True} do {\Skip} } : \True => \False$
is valid but not derivable.
Thus by definition, the proof rules are incomplete.
\end{proof}



\subsection{Admissible Rules}

\begin{conjecture}
    The following rule is acyclically admissible w.r.t.\
    the rules in \ref{fig:unary-hl-rules}:
    \[ \inferrule*[right=HL-Case]{
        \Triple c : P \wedge e => Q \\
        \Triple c : P \wedge \neg e => Q
    }{
        \Triple c : P => Q
    }
    \]
\end{conjecture}

\section{Rule graphs}

\newcommand*{\DomLiftable}{\mathit{Lift}}
\newcommand*{\DomStmt}{\mathit{Stmt}}
\newcommand*{\DomRule}{\mathit{Rule}}
\newcommand*{\LIFT}{\textsc{Lift}}

Suppose we have a family of statements $\DomStmt$,
where for each statement $S \in \DomStmt$
we have a defined notion of validity,
denoted $\VAL S$.
%
We also have a subset of \emph{liftable} statements
$\DomLiftable \subseteq \DomStmt$;
these are statements for which
we are ``comfortable'' lifting a proof of their derivability
to a metalogical proof of their validity.
%
We also have a family of rules $\DomRule$ of the form
\[
    \inferrule*[right=Rule-Name]{
        \DER S_1 \\ \cdots \\ \DER S_n
    }{
        \DER C
    }
\]
%
\begin{definition}
    A rule with premises $S_1,\dots,S_n$ and conclusion $C$ is \emph{locally sound} if
    \[\VAL S_1 \wedge \cdots \wedge \VAL S_n \implies \VAL C.\]
\end{definition}
%

\newcommand*{\DomNode}{\mathit{Nodes}}
\newcommand*{\node}{\mathit{node}}
\newcommand*{\MapConc}{\mathit{conc}}
\newcommand*{\MapRule}{\mathit{rule}}
\newcommand*{\MapPrems}{\mathit{prems}}

\begin{definition}
A \emph{rule graph} is a tuple $G=(\DomNode,\MapConc,\MapRule,\MapPrems)$ for a
finite set of node names $\DomNode$,
and mappings %from node names to statements 
    $\MapConc : \DomNode \ra \DomStmt$,
%a mapping from node names to rule names 
    $\MapRule : \DomNode \ra \DomRule + \LIFT$,
%and a mapping from node names to premises 
 and   $\MapPrems : \DomNode \ra \DomNode^*$,
such that for every $\node \in \DomNode$:
\begin{itemize}
\item If $\MapRule(\node) = r \in \DomRule$
    where $\abs{p} = n$, then
    \[
        \inferrule{
            \DER \MapConc(\MapPrems_1(\node))
            \\ \cdots \\
            \DER \MapConc(\MapPrems_n(\node))
        }{ \DER \MapConc(\node) }
    \]
    is a syntactically valid instance of $r$,\\
    where $\MapPrems_i(\node)$
    denotes $\pi_i(\MapPrems(\node))$.
\item Otherwise, if $\MapRule(\node) = \LIFT$,
    then $\MapPrems(\node) = []$ and 
    $\MapConc(\node) \in \DomLiftable$.
\end{itemize}
\end{definition}

A statement is \emph{in} a graph if $S \in \mathrm{Im}(G.\MapConc)$.
Denote with $S \in G$.

\begin{definition}
    A \emph{path} in a rule graph $(\DomNode,\MapConc,\MapRule,\MapPrems)$
    is a possibly infinite sequence of nodes
    $\node_1,\node_2,\dots$
    all in $\DomNode$
    such that
    $\forall i > 0.\
    \node_{i+1} \in \MapPrems(\node_i).
    $
\end{definition}

\newcommand*{\reaches}[3]{%
    \ensuremath{#1 \leadsto_{#3} #2}%
}
For a rule graph $G$ with a node set $\DomNode$,
and a pair of nodes $\node, \node' \in \DomNode$,
we will let \reaches{\node}{\node'}{G} denote
the existence of a path in $G$
that begins with $\node$ and ends with $\node'$.
Note how a path containing a liftable node 
which employs the \LIFT\ rule must end at that node.

\begin{definition}
    A \emph{(simple) cycle} is a finite path $\node_1,\dots,\node_n$
    for $n>1$,
    where $\node_1 = \node_n$ and all other nodes are distinct.
    %
    A rule graph is \emph{cyclic}
    iff it contains a path which is a cycle.
\end{definition}


\begin{lemma}
    A rule graph is cyclic iff
    it does not have a longest path.
\end{lemma}
\begin{proof}
First direction:
Consider any path in the graph.
We can always produce a longer path by iterating over the graph's cycle a sufficient number of times.

Second direction:
If there is no longest path, then there is a path longer than the cardinality of $\DomNode$. By pigeonhole w.r.t.\ the finitude of $\DomNode$,
this path must contain a duplicate node.
A path containing a duplicate node must contain a cycle.
%See Coq. (Depends on finitude of $\DomNode$)
\end{proof}

\newcommand{\restrict}[2]{%
    #1 {\mid_{#2}} %
}

\iffalse
\begin{definition}
    The \emph{subgraph} of a graph $G=(\DomNode,\MapConc,\MapRule,\MapPrems)$ w.r.t.\ $\node \in \DomNode$
    is 
    \\
    $G'=(
        \DomNode',\
        \restrict\MapConc{\DomNode'},\
        \restrict\MapRule{\DomNode'},\
        \restrict\MapPrems{\DomNode'}
    )$,
    where $\DomNode' = 
        \SetBuild(\node' \in \DomNode | \reaches{\node}{\node'}{G})$.
\end{definition}

\begin{lemma}
Any subgraph is well-defined according to the definition of graph.
\end{lemma}
\fi

\subsection{Soundness of acyclic graphs}

\begin{definition}
$(\DomNode,\MapConc,\MapRule,\_)$ is \emph{lift-valid}
if
\(
    \forall \node \in \DomNode.\ \MapRule(\node) = \LIFT \implies
    \VAL \MapConc(\node).
\)
\end{definition}

\begin{definition}
    A graph is \emph{locally sound} if
    all rules in the graph are locally sound.
\end{definition}

\begin{lemma}
    \label{lem:acyclic-soundness}
    If a graph $G$ is lift-valid, locally sound, and acyclic,
    then $\VAL S$ for every $S \in G$.
    %
    %Consider a graph $G=(\DomNode,\MapConc,\MapRule,\MapPrems)$.
    %and a node $\node \in \DomNode$.
    %If $G$ is acyclic, lift-valid, and locally sound,
    %and all rules in the image of $\MapRule$ are locally sound,
    %then $\VAL \MapConc(\node)$ for every node $\node \in \DomNode$.
\end{lemma}

\begin{proof}
We induct over the depth of nodes in the rule graph.
The depth of a leaf is 1;
the depth of a non-leaf is one greater than the maximum depth of all premises.
From local soundness, we know that the validity of a conclusion
comes from the validity of its premises. 
We also know that the depth of a conclusion is greater than that of its premises.
Thus by strong induction on depth,
where the base case is every leaf
(which is either valid by lack of premises or by being lifted),
we find that every statement in the graph must be valid.

\end{proof}

Consider how depth is only well-defined for an acyclic graph.
This means computing depth in our Coq implementation required
proving that depth always exists as a finite value for each node in an acyclic graph.
We implemented depth as a fixpoint with fuel, where the initial fuel is the cardinality of $\DomNode$.
We then had to show that this amount of fuel is sufficient. We argue that if an amount of fuel $f$ is insufficient for a node $n$,
then a path from $n$ with length $> f$ must exist.
Thus, if $\abs\DomNode$ is insufficient fuel for a node,
then a path longer than $\abs\DomNode$ exists,
which is impossible in an acyclic graph.

\iffalse
\begin{lemma}
    Consider a graph $G=(\DomNode,\MapConc,\MapRule,\MapPrems)$,
    and a node $\node \in \DomNode$.
    If the subgraph of $G$ w.r.t.\ $\node$ is acyclic and valid,
    and all rules in the image of $\MapRule$ are locally sound,
    then $\VAL \MapConc(\node)$.
\end{lemma}
\fi

\newcommand{\DERby}[2]{\DER_{#1}{#2}}

\begin{definition}
    $S$ is \emph{derivable via graph} $G$, denoted $\DERby G S$,
    if $S \in G$ and $G$ is lift-valid.\\
    $S$ is \emph{derivable}, denoted $\DER S$,
    if a graph $G$ exists s.t.\ $\DERby G S$.
    %if a lift-valid graph $(\_, conc, \_, \_)$ exists
    %which is lift-valid, and
    %where $S \in \text{Im}(conc)$.
    %contains $\node$
    %where $\MapConc(\node) = S$.
\end{definition}

\begin{theorem}
    For a statement $S$,
    if $\DERby G S$ for an acyclic and locally sound graph $G$, %valid acyclic graph,
    %for which all contained rules are locally sound,
    then $\VAL S$.
\end{theorem}

\begin{proof}
    Follows from Lemma~\ref{lem:acyclic-soundness}.
\end{proof}


\subsection{Completeness}

\begin{definition}
    Suppose we have an oracle $O$ s.t.\
    for any $S \in \DomLiftable$,
    $O$ accepts $S$ iff $\VAL S$.
    A proof system is \emph{relatively complete} w.r.t.\
    the completeness of the metalogical proofs of lifted statements
    %$\DomLiftable$ if,
    if, for any $S \in \DomStmt$ where $\VAL S$,
    a locally sound $G$ exists where $S \in G$
    and $O$ accepts every lifted statement in $G$.
\end{definition}

\subsection{Other stuff}

\begin{definition}
    A graph $(\DomNode,\MapConc,\MapRule,\MapPrems)$
    is \emph{normal} if
    $\MapConc$ is injective.
\end{definition}

Normalization procedure for DAG:
\begin{enumerate}[label=(\roman*)]
    \item Topologically sort the DAG.
    \item If every node is unique, halt. Otherwise:
    \item Consider node $n$ for which there are (at least) two instances.
        Take all edges pointing into the earlier $n$ and redirect them into the latter $n$.
    \item While nodes of indegree 0 exist, remove them.
    \item Return to step (ii).
\end{enumerate}

\begin{conjecture}
If 
\end{conjecture}
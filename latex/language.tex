
\section{Operational semantics of the \textsc{While} language}

\subsection*{Syntax}

\begin{displaymath}
\begin{array}{lrcl}
    & n & \in & \Z \\
    & x & \in  & \domvar \\
    \domvar & v & \bnf & n \ALT \True \ALT \False \\
    \DomAexp & a & \bnf & x \ALT n \ALT a + a \ALT a - a \ALT a \times a \\
    \DomBexp & b & \bnf & \True \ALT \False \ALT 
        \neg b \ALT b \wedge b \ALT b \vee b \ALT b \Rightarrow b \\\
    && \ALT & a = a \ALT b = b \ALT a < a \ALT a \leq a \\
    & e & \bnf & a \ALT b \\ 
    \DomComm & c & \bnf & \Skip \ALT \Assn x = a
        \ALT \Seq c ; c \\
    && | & \ITE b then c else c \\
    && | & \While b do c
\end{array}
\end{displaymath}

\subsection*{Memories and expressions}

Memories (in the family $\Mems$)
are total maps from variables (\domvar) to integers.
%
We define the \emph{evaluation} of an expression $e$
w.r.t.\ a memory $m$, denoted $\Eval(e,m)$,
to yield exactly one value $v \in \Vals$.
%
{
\newcommand{\LINE}[2]{
    \Eval(#1,m) &\triangleq& #2 \\
}
%
\begin{mathpar}
\begin{array}{rcl}
\LINE x {m(x)}
\LINE n n
\LINE \True \True
\LINE \False \False
\LINE {\neg b} 
    {\neg \Eval(b,m)}
\LINE {b_1 \wedge b_2} {
    \Eval(b_1,m) \wedge \Eval(b_2,m)}
\LINE {b_1 \vee b_2} {
    \Eval(b_1,m) \vee \Eval(b_2,m)}
\LINE {b_1 \Rightarrow b_2} {
    \Eval(b_1,m) \Rightarrow \Eval(b_2,m)}
\end{array}
\and
\begin{array}{rcl}
\LINE {a_1 = a_2} {
    \Eval(a_1,m) = \Eval(a_2,m)}
\LINE {b_1 = b_2} {
    \Eval(b_1,m) = \Eval(b_2,m)}
\LINE {a_1 < a_2} {
    \Eval(a_1,m) < \Eval(a_2,m)}
\LINE {a_1 \leq a_2} {
    \Eval(a_1,m) \leq \Eval(a_2,m)}
\LINE {a_1 + a_2} {
    \Eval(a_1,m) + \Eval(a_2,m)}
\LINE {a_1 - a_2} {
    \Eval(a_1,m) - \Eval(a_2,m)}
\LINE {a_1 \times a_2} {
    \Eval(a_1,m) \times \Eval(a_2,m)}
\end{array}
\end{mathpar}

}

%Define the support of an expression $e$ as
%$\Support e \triangleq \SetBuild(m | {\Eval(e,m) \neq \bot})$.

%A \emph{boolean expression} is an expression $P$ s.t.
%$\forall m.\ \Eval(P,m) \in \set{\True,\False,\bot}$.
%Call this family of expressions $\BoolExps$.
%

% \[
%     \begin{prooftree}[center=false]
%     \hypo{\Eval(e,m)=v}
%     \infer1{\Cfg(\Assn x = e, m) \sstepsto \Cfg(\Skip, m)}
%     \end{prooftree}
%     \qquad
%     \begin{prooftree}[center=false]
%     \hypo{ \Gamma, A \vdash B }
%     \infer1{ \Gamma \vdash A \to B }
%     \end{prooftree}
% \]


% \subsection{Memories}

\subsection*{Small-step operational semantics}

\begin{mathpar}
    % Assignment
    \inferrule*[right=SS-Assn]
    { \Eval(a,m)=v }
    { \Cfg(\Assn x = a, m) \sstepsto \Cfg(\Skip, \MemDef m[x->v]) }
    % Sequencing
    \\ \inferrule*[right=SS-Seq]
    { \Cfg(c_0,m) \sstepsto \Cfg(c'_0,m') }
    { \Cfg(\Seq c_0 ; c_1, m) \sstepsto \Cfg(c'_0 ; c_1, m')}
    % Sequencing, skip elimination
    \and \inferrule*[right=SS-Seq-Skip]
    { }
    { \Cfg(\Seq \Skip ; c, m) \sstepsto \Cfg(c, m) }
    % If true
    \and \inferrule*[right=SS-If-T]
    { \Eval(b,m) = \True}
    { \Cfg(\ITE b then c_1 else c_2, m) \sstepsto \Cfg(c_1, m)}
    % If false
    \and \inferrule*[right=SS-If-F]
    { \Eval(b,m) = \False}
    { \Cfg(\ITE b then c_1 else c_2, m) \sstepsto \Cfg(c_2, m)}
    % While true
    \and \inferrule*[right=SS-While-T]
    { \Eval(b,m) = \True}
    { \Cfg(\While b do c, m) \sstepsto \Cfg(\Seq c; \While b do c, m)}
    % While false
    \and \inferrule*[right=SS-While-F]
    { \Eval(b,m) = \False}
    { \Cfg(\While b do c, m) \sstepsto \Cfg(\Skip, m)}
\end{mathpar}

\subsection*{Multi-step reduction}

$\Cfg(c,m) \sstepston n \Cfg(c',m')$ iff
$\Cfg(c_0,m_0),\dots,\Cfg(c_n,m_n)$
s.t.\ $\Cfg(c,m)=\Cfg(c_0,m_0)$,
$\Cfg(c_k,m_k)=\Cfg(c',m')$,
and $\forall i \in [0,n).\ \Cfg(c_i,m_i) \sstepsto \Cfg(c_{i+1},m_{i+1})$.


Let $\sstepstotrans$ be the reflexive transitive closure of $\sstepsto$. Namely,
$\Cfg(c,m) \sstepstotrans \Cfg(c',m')$ iff
$\exists n.\ \Cfg(c,m) \sstepston n \Cfg(c',m')$.

\begin{definition}
    $c$ \emph{yields} $m'$ from $m$,
    denoted $\Yields(c,m,m')$,
    if $\Cfg(c,m) \sstepstotrans \Cfg(\Skip,m')$.
    $c$ \emph{terminates} on $m$, denoted $\Terminate(c,m)$,
    iff $\exists m'. \Yields(c,m,m')$.
    $c$ \emph{diverges} on $m$, denoted $\Diverge(c,m)$,
    iff $\neg \Terminate(c,m)$.
\end{definition}

\begin{lemma}[Sequential execution]
    If $\Cfg(c_1; c_2, m) \sstepston n \Cfg(\Skip, m'')$, 
    then intermediate memory $m'$ exists s.t.\
    $\Cfg(c_1, m) \sstepston{n_1} \Cfg(\Skip, m')$
    and $\Cfg(c_2, m') \sstepston{n_2} \Cfg(\Skip, m'')$
    where $n_1+n_2+1=n$.
\end{lemma}
\begin{proof}
    By induction on length of execution. See Coq proof.
\end{proof}

\begin{lemma}
    If $\Cfg(c,m) \sstepston n \Cfg(c_1,m_1)$,
    and $\Cfg(c,m) \sstepston n \Cfg(c_2,m_2)$,
    then $c_1 = c_2$ and $m_1 = m_2$.
\end{lemma}

\begin{lemma}
    If $\Cfg(c,m) \sstepston{n_1} \Cfg(\Skip,m_1)$,
    and $\Cfg(c,m) \sstepston{n_2} \Cfg(\Skip,m_2)$,
    then $m_1 = m_2$ and $n_1 = n_2$.
\end{lemma}

\section{Assertions}

\newcommand*{\domivar}{\mathsf{IVar}}

\newcommand*{\DomAssert}{\mathsf{Assrt}}
\newcommand*{\DomAexpv}{\mathsf{Aexpv}}

We define boolean assertions as an extension
on boolean expressions.
These allow for quantification over
integer variables.
\begin{displaymath}
\begin{array}{lrcl}
    & i & \in  & \domivar \\
    \DomAexpv & a & \bnf & x \ALT n \ALT i \ALT a + a \ALT a - a \ALT a \times a \\
    \DomAssert & b & \bnf & \True \ALT \False \ALT 
        \neg b \ALT b \wedge b \ALT b \vee b \ALT b \Longrightarrow b \\\
    && \ALT & a = a \ALT b = b \ALT a < a \ALT a \leq a \\
    && \ALT & \forall i.\, b \ALT \exists i.\, b
\end{array}
\end{displaymath}
where $i$ are bound integer variables.
Define a \emph{binding} $I$ as a total mapping $\domivar \rightarrow \Z$.
%
Note how we may implicitly lift any
arithmetic expression $a \in \DomAexp$ into
an expression in $\DomAexpv$,
and similarly a
boolean expression $b \in \DomBexp$ into
an assertion in $\DomAssert$, which
contains no integer variables.

\def\AVAL#1|#2|-#3{%
    #1 \vDash_{#2} #3 %
}
\def\NAVAL#1|#2|-#3{%
    #1 \nvDash_{#2} #3 %
}

\def\AVALmi#1{%
    \AVAL m|I|- #1 %
}

\def\AVALU#1{%
    \vDash #1 %
}

\def\EvalI(#1,#2,#3){%
    \Interp{#1}_{#2,#3}
}

Define evaluation of arithmetic expressions with integer variables:
%
{
\newcommand{\LINE}[2]{
    \EvalI(#1,m,I) &\triangleq& #2 \\
}
%
\begin{mathpar}
\begin{array}{rcl}
\LINE {n} {
    n}
\LINE {x} {
    m(x)}
\LINE {i} {
    I(i)}
\end{array}
\and
\begin{array}{rcl}
\LINE {a_1 + a_2} {
    \EvalI(a_1,m,I) + \EvalI(a_2,m,I)}
\LINE {a_1 - a_2} {
    \EvalI(a_1,m,I) - \EvalI(a_2,m,I)}
\LINE {a_1 \times a_2} {
    \EvalI(a_1,m,I) \times \Eval(a_2,m,I)}
\end{array}
\end{mathpar}
}


The relation $\AVAL m|I|-P$
says that memory $m$ \emph{satisfies} assertion $P$
with binding $I$.
Notation $\AVALU P$ says that $P$ always holds
in any memory and binding.
The inductive rules for \emph{satisfaction} are:
%
\begin{mathpar}
\inferrule*[right=Sat-\True]{ }{
    \AVALmi \True
}
\and
% \inferrule*[right=Val-\False]{ }{
%     \NAVAL m|I|- \False
% }
% \and 
\inferrule*[right=Sat-Neg]{
    \neg\ \AVALmi P
}{
    \AVALmi \neg P
}
\and
\inferrule*[right=Sat-$\wedge$]{
    \AVALmi P \\
    \AVALmi Q
}{
    \AVALmi P \wedge Q
}
\and \\
\inferrule*[right=Sat-$\vee_L$]{
    \AVALmi P
}{
    \AVALmi P \vee Q
}
\and
\inferrule*[right=Sat-$\vee_R$]{
    \AVALmi Q
}{
    \AVALmi P \vee Q
}
\and
\inferrule*[right=Sat-$\Rightarrow$]{
    \AVAL m|I|- P \Longrightarrow \AVAL m|I|- Q
}{
    \AVALmi P \Longrightarrow Q
}
\and
% \inferrule*[right=Val-$\Rightarrow_L$]{
%     \NAVAL m|I|- P
% }{
%     \AVALmi P \Longrightarrow Q
% }
% \and
% \inferrule*[right=Val-$\Rightarrow_R$]{
%     \AVALmi Q
% }{
%     \AVALmi P \Longrightarrow Q
% }
% \and
\inferrule*[right=Sat-$\forall$]{
    \forall n.\, \AVAL m| I[i \mapsto n] |- P
}{
    \AVALmi \forall i.\, P
}
\and
\inferrule*[right=Sat-$\exists$]{
    %\exists n.\, 
    \AVAL m| I[i \mapsto n] |- P
}{
    \AVALmi \exists i.\, P
}
\and
\inferrule*[right={Sat-$=_A$}]{
    \EvalI(a_1,m,I) = \EvalI(a_2,m,I)
}{
    \AVALmi a_1 = a_2
}
\and
\inferrule*[right=Sat-$<$]{
    \EvalI(a_1,m,I) < \EvalI(a_2,m,I)
}{
    \AVALmi a_1 < a_2
}
\and
\inferrule*[right=Sat-$\leq$]{
    \EvalI(a_1,m,I) \leq \EvalI(a_2,m,I)
}{
    \AVALmi a_1 \leq a_2
}
\and
\inferrule*[right={Sat-$=_B$}]{
    \AVALmi b_1 \iff \AVALmi b_2
}{
    \AVALmi b_1 = b_2
}
\end{mathpar}

Note that in our Coq implementation,
rather than having individual rules
we assert satisfaction
by recursively translating an assertion
into a Coq-level proposition.
%
We then prove the above rules hold
w.r.t.\ this translation.

We prove various properties of assertions and expressions.
Note that when we discuss substitution into boolean expressions,
substitutions only affect free variables ---
they are overridden by quantified variables.

\begin{lemma}
For $b \in \DomBexp$,
$\AVALmi b \iff \Eval(b,m) = \True$.
\end{lemma}

\begin{lemma}
For $n \in \Z$ and $a \in \DomAexpv$,
$\EvalI(a[n/x],m,I) = \EvalI(a,m[x \mapsto n],I)$.
Similarly,
$\EvalI(a[n/i],m,I) = \EvalI(a,m,I[i \mapsto n])$.
\end{lemma}

\begin{lemma}
For $n \in \Z$,
$\AVAL m | I |- P[n/x] \iff \AVAL m[x \mapsto n] | I |- P$.
Similarly,
$\AVAL m | I |- P[n/i] \iff \AVAL m | I[i \mapsto n] |- P$.
\end{lemma}

\begin{corollary}
$\AVALmi \forall i.\, P \iff \forall n.\, \AVALmi P[n/i]$.
Similarly,
$\AVALmi \exists i.\, P \iff \exists n.\, \AVALmi P[n/i]$.
\end{corollary}

\iffalse
We define the \emph{interpretation} of $P,Q \in \DomAssert$
to yield the subset of memories
in which the assertion holds:
$\Interp P \triangleq \SetBuild(m | {\Eval(P,m) = \True})$.
Properties of interpretation include (all proven in Coq):
%
\begin{mathpar}
\begin{array}{rcl}
\Interp{\neg P} & = & \complement{\Interp{P}} \\
\Interp{P \wedge Q} & = & \Interp{P} \cap \Interp{Q} \\
\Interp{P \vee Q} & = & \Interp{P} \cup \Interp{Q}
\end{array}
\and
\begin{array}{rcl}
\Interp{P \Rightarrow Q} & = & \complement{\Interp P} \cup \Interp{Q} \\
\Interp{\ETrue} & = & \Mems  \\
\Interp{\EFalse} & = & \emptyset
\end{array}
\end{mathpar}

A boolean expression is \emph{valid},
denoted $\VAL P$,
if $P$ always yields $\True$,
i.e.\ $\Interp P = \Mems$.
Properties include:
{
%
\newcommand*{\LINE}[3]{%
    \VAL #1 & \text{#3} & #2 \\ %
}
%
\begin{mathpar}
{\BIIMP\inferrule*[right=Val-$\wedge$]{
    \VAL P \\ \VAL Q
}{
    \VAL P \wedge Q
}}
%
\and
%
\inferrule*[right=Val-$\vee_\textsc{L}$]{
    \VAL P
}{
    \VAL P \vee Q
}
%
\and
%
\inferrule*[right=Val-$\vee_\textsc{R}$]{
    \VAL Q
}{
    \VAL P \vee Q
}
%
\and
{\BIIMP\inferrule*[right=Val-$\Rightarrow$]{
    \Interp P \subseteq \Interp Q
}{
    \VAL P \Rightarrow Q
}}
\end{mathpar}
}
\fi

